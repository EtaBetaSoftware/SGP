\documentclass[a4paper,10pt,twoside]{book}

%*******************************************************************************
% Codifica e lingua
%*******************************************************************************
\usepackage[utf8x]{inputenc}
\usepackage[T1]{fontenc}
\usepackage[english,italian]{babel}
\usepackage{lmodern}
\usepackage{multirow}
\usepackage{amsmath,eurosym}
%*******************************************************************************
% Qualche macro utile a tutti
%*******************************************************************************
\newcommand{\docRoot}{..}
\newcommand{\inglese}[1]{\foreignlanguage{english}{\textit{#1}}}
\newcommand{\team}{\textsf{EtaBeta\,Software}\xspace}
\newcommand{\caName}{BPM-1.0\xspace}
\newcommand{\customer}{\textsf{Alpha\,\&\,Partners}\xspace}
\newcommand{\mktg}{\inglese{marketing}\xspace}
\newcommand{\bsn}{\inglese{business}\xspace}
\newcommand{\sw}{\inglese{software}\xspace}
\newcommand{\tick}{\textcolor{green}{\ding{52}}}
\newcommand{\cross}{\textcolor{red}{\ding{56}}}
\newcommand{\swname}[1]{\textsf{#1}}

%*******************************************************************************
% Figure e immagini
%*******************************************************************************
\usepackage{graphicx}
\graphicspath{{\docRoot/shared/pictures/}}
\usepackage{float}

%*******************************************************************************
% Tabelle
%*******************************************************************************
\usepackage{booktabs}
\usepackage{array}
\usepackage{longtable}

%*******************************************************************************
% Elenchi puntati personalizzati
%*******************************************************************************
\usepackage{enumitem}

%*******************************************************************************
% Landscape per ruotare le pagine
%*******************************************************************************
\usepackage{pdflscape}

%*************************************************
% Collegamenti intra- e intertestuali
%*************************************************
\usepackage{hyperref}
\hypersetup{%
    colorlinks=false,linktocpage=false,pdfborder={0 0 0},%
    pdfstartpage=1, pdfstartview=FitV,plainpages=false,%
    urlcolor=Black, linkcolor=Black,
    pdfcreator={pdfLaTeX},%
    pdfproducer={pdfLaTeX with hyperref package}%
}

%*************************************************
% Grafica vettoriale
%*************************************************
\usepackage{tikz}
\usetikzlibrary{shadows,arrows,shapes}

%*******************************************************************************
% Altri pacchetti
%*******************************************************************************
\usepackage{xspace} % per spazi condizionali extra
\usepackage{lastpage} % per sapere il numero totale di pagine
\usepackage{microtype} % qualche accorgimento tipografico
\usepackage{pifont} % per avere alcuni simboli
\usepackage{ifthen} % per scelte condizionali

% *************************************************
% Intestazioni e piè di pagina personalizzati
% *************************************************
\usepackage{fancyhdr}

% stile normale
\fancypagestyle{normal}{
  \fancyhead{}
  \fancyhead[RE,LO]{\sffamily\team}
  \fancyhead[LE,RO]{\sffamily\leftmark}
  \renewcommand{\headrulewidth}{.4pt}
  \fancyfoot{}
  \fancyfoot[LE,RO]{\sffamily{}pag. \thepage{} di \pageref{LastPage}}
  \fancyfoot[RE,LO]{\sffamily\doctitle}
  \renewcommand{\footrulewidth}{.4pt}
}

% stile per gli indici
\fancypagestyle{toc}{
  \fancyhead{}
  \renewcommand{\headrulewidth}{0pt}
  \fancyfoot{}
  \fancyfoot[RO,LE]{\sffamily\thepage{}}
  \fancyfoot[RE,LO]{\sffamily\doctitle{}}
  \renewcommand{\footrulewidth}{.4pt}
}

% stile per le prime pagine dei capitoli
\fancypagestyle{plain}{
  \fancyhead{}
  \renewcommand{\headrulewidth}{0pt}
  \fancyfoot{}
  \fancyfoot[RO,LE]{\sffamily\thepage{} di \pageref{LastPage}}
  \fancyfoot[RE,LO]{\sffamily\doctitle}
}

\pagestyle{fancy}
\renewcommand{\sectionmark}[1]{\markboth{#1}{#1}}
