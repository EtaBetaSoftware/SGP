%*******************************************************************************
% Macro per il documento corrente
%*******************************************************************************
\newcommand{\sharedPath}{../shared}
\newcommand{\doctitle}{Piano di Progetto}


%*******************************************************************************
% Preambolo
%*******************************************************************************
\documentclass[a4paper,10pt,twoside]{article}

%*******************************************************************************
% Codifica e lingua
%*******************************************************************************
\usepackage[utf8x]{inputenc}
\usepackage[T1]{fontenc}
\usepackage[english,italian]{babel}

%*******************************************************************************
% Qualche macro utile a tutti
%*******************************************************************************
\newcommand{\docRoot}{..}
\newcommand{\inglese}[1]{\foreignlanguage{english}{\textit{#1}}}
\newcommand{\team}{EtaBeta Software\xspace}
\newcommand{\caName}{BPM-1.0\xspace}

%*******************************************************************************
% Figure e immagini
%*******************************************************************************
\usepackage{graphicx}
\graphicspath{{\docRoot/shared/pictures/}}

%*******************************************************************************
% Tabelle
%*******************************************************************************
\usepackage{booktabs}

%*******************************************************************************
% Elenchi puntati personalizzati
%*******************************************************************************
\usepackage{enumitem}

%*************************************************
% Collegamenti intra- e intertestuali
%*************************************************
\usepackage{hyperref}
\hypersetup{%
    colorlinks=false,linktocpage=false,pdfborder={0 0 0},%
    pdfstartpage=1, pdfstartview=FitV,plainpages=false,%
    urlcolor=Black, linkcolor=Black,
    pdfcreator={pdfLaTeX},%
    pdfproducer={pdfLaTeX with hyperref package}%
}

% **************************************************
% Definizione geometria della pagina
% **************************************************
\usepackage[a4paper,head=4cm,top=4.5cm,bottom=3cm,left=3cm,right=3cm,bindingoffset=5mm]{geometry}

%*******************************************************************************
% Altri pacchetti
%*******************************************************************************
\usepackage{xspace} % per spazi condizionali extra
\usepackage{lastpage} % per sapere il numero totale di pagine

% *************************************************
% Intestazioni e piè di pagina personalizzati
% *************************************************
\usepackage{fancyhdr}

% stile normale
\fancypagestyle{normal}{
\fancyhead{}
\fancyhead[LE,RO]{
\sffamily\team
}
\fancyhead[RE,LO]{
\sffamily\leftmark
}
\renewcommand{\headrulewidth}{.4pt}
\cfoot{}
\fancyfoot[RO,LE]{\sffamily
  pag. \thepage{} di \pageref{LastPage}}
\fancyfoot[RE,LO]{\sffamily\doctitle}
\renewcommand{\footrulewidth}{.4pt}
}

% stile per gli indici
\fancypagestyle{toc}{
\fancyhead{}
\fancyhead[LE,RO]{
\sffamily\team
}
\fancyhead[RE,LO]{
\sffamily\caName
}
\renewcommand{\headrulewidth}{.4pt}
\cfoot{}
\fancyfoot[RO,LE]{\sffamily\thepage{}}
\fancyfoot[RE,LO]{\sffamily\doctitle{}}
\renewcommand{\footrulewidth}{.4pt}
}

\pagestyle{fancy}
\renewcommand{\sectionmark}[1]{\markboth{#1}{#1}}


%*******************************************************************************
% Inizio documento
%*******************************************************************************
\begin{document}

\pagestyle{empty}
\begin{center}

{\sffamily
Sviluppo e Gestione Progetti\\
a.a. 2012--2013
}

\vskip 1.5cm

\includegraphics[width=\textwidth]{logo}

\medskip
{\Huge\sffamily\bfseries
\team
}

\vskip 1.5cm

% titolo del progetto
{\Large\sffamily\bfseries
\caName
}

\vskip 1cm

% titolo del documento
\hrule
\vskip 10pt
{\Huge\scshape
\doctitle
}
\vskip 10pt
\hrule

\end{center}

\clearpage

\tableofcontents{\thispagestyle{toc}}

\clearpage

\pagestyle{normal}
\pagenumbering{arabic}




\section{Introduzione}


\subsection{Scopo del documento}

\clearpage


\section{Work Breakdown Structure}

La Work Breakdown Structure (WBS) costituisce il primo passo verso una buona pianificazione.

La WBS sfrutta il concetto di \textit{``Divide ed Impera''}; essa infatti ha lo scopo di suddividere il lavoro di progetto in parti minori: in questo modo anche i progetti più complessi diventano realizzabili.

La scomposizione avviene in modo gerarchico: il lavoro viene suddiviso in attività che a loro volta vengono scomposte in sotto attività più piccole e semplici da gestire, dette Work Package (WP).
La rappresentazione della scomposizione avviene tramite l'uso di un albero, dove i nodi sono le attività e le foglie i WP.

La WBS permette, nella suddivisione del lavoro in pezzi minori, di analizzare tutte le implicazioni del progetto senza tralasciare nulla, inoltre costituisce anche un buon punto di partenza per calcolare costi e benefici del progetto.
La WBS offre una chiara visione del prodotto finale e del processo complessivo attraverso il quale è stato creato il prodotto.


Si presenta di seguito la definizione della WBS per il progetto ....%TODO questo va scritto alla fine, non si può fuffare ora!
Ogni sezione riporta la descrizione dell'attività di primo livello e la descrizione relativa ad ogni suo WP. 

\subsection{WBS di progetto}

Il team ha scelto di strutturare la WBS seguendo il ciclo di vita del progetto.
%\begin{figure}[h!]
%  \includegraphics[width=\textwidth]{}
%\caption{Work Breakdown Structure}
%\end{figure}

\subsubsection{Pianificazione}
In questa fase verranno eseguite tutte le attività necessarie alla pianificazione del progetto.
Dovranno essere pianificate le attività da svolgere, i tempi e le risorse necessarie. Dovrà inoltre essere redatto il Business  Plan.
\subsubsection{Analisi di Mercato e Studio di Fattibilità}
Questo WP prevede uno studio generale del mercato sui /inglese{software} BPM, con lo scopo di permettere l'organizzazione della pianificazione in base alle informazioni tratte dagli studio.
In particolare, lo studio di fattibilità dà al team la possibilità di capire se gli strumenti e le risorse di cui dispone sono sufficienti a garantire il conseguimento dell'obbiettivo.
L'analisi, invece, permetterà di eseguire una corretta pianificazione sulla base di informazioni vere riguardanti il mercato attuale. 
\subsubsection{Pianificazione del Lavoro}
\subsubsection{Redazione Business Plan}


\subsubsection{Studio di Mercato}
\subsubsection{Proposta}

\section{Organizational Breakdown Structure}
l'Organizational Breakdown Structure (OBS) rappresenta l'organizzazione del progetto rispetto alle risorse umane impiegate in esso.
L'OBS rappresenta una scomposizione gerarchica delle responsabilità di progetto, generata alla scopo di individuare e responsabili dei WP. 
L'OBS deve essere creata in seguito alla redazione della WBS. Infatti, solo dopo aver tracciato i WP è possibile assegnare loro un responsabile/esecutore.
La creazione della OBS risulta utile sia sotto l'aspetto gerarchico in quanto permette al \inglese{Project Manger} di individuare i responsabili, sia sotto l'aspetto organizzativo degli esecutori materiali delle attività, in quanto facilita la comunicazione tra essi permettendo di capire a chi chiedere cosa. 

\section{Matrice delle Responsabilità}
\section{Resource Breakdown Structure}
\section{Pianificazione Temporale}
\end{document}