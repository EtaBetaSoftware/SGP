%*******************************************************************************
% Macro per il documento corrente
%*******************************************************************************
\newcommand{\sharedPath}{../shared}
\newcommand{\doctitle}{Progetto SGP}
\newcommand{\docauthor}{\team}

%*******************************************************************************
% Preambolo
%*******************************************************************************
\documentclass[a4paper,10pt,twoside]{article}

%*******************************************************************************
% Codifica e lingua
%*******************************************************************************
\usepackage[utf8x]{inputenc}
\usepackage[T1]{fontenc}
\usepackage[english,italian]{babel}

%*******************************************************************************
% Qualche macro utile a tutti
%*******************************************************************************
\newcommand{\docRoot}{..}
\newcommand{\inglese}[1]{\foreignlanguage{english}{\textit{#1}}}
\newcommand{\team}{EtaBeta Software\xspace}
\newcommand{\caName}{BPM-1.0\xspace}

%*******************************************************************************
% Figure e immagini
%*******************************************************************************
\usepackage{graphicx}
\graphicspath{{\docRoot/shared/pictures/}}

%*******************************************************************************
% Tabelle
%*******************************************************************************
\usepackage{booktabs}

%*******************************************************************************
% Elenchi puntati personalizzati
%*******************************************************************************
\usepackage{enumitem}

%*************************************************
% Collegamenti intra- e intertestuali
%*************************************************
\usepackage{hyperref}
\hypersetup{%
    colorlinks=false,linktocpage=false,pdfborder={0 0 0},%
    pdfstartpage=1, pdfstartview=FitV,plainpages=false,%
    urlcolor=Black, linkcolor=Black,
    pdfcreator={pdfLaTeX},%
    pdfproducer={pdfLaTeX with hyperref package}%
}

% **************************************************
% Definizione geometria della pagina
% **************************************************
\usepackage[a4paper,head=4cm,top=4.5cm,bottom=3cm,left=3cm,right=3cm,bindingoffset=5mm]{geometry}

%*******************************************************************************
% Altri pacchetti
%*******************************************************************************
\usepackage{xspace} % per spazi condizionali extra
\usepackage{lastpage} % per sapere il numero totale di pagine

% *************************************************
% Intestazioni e piè di pagina personalizzati
% *************************************************
\usepackage{fancyhdr}

% stile normale
\fancypagestyle{normal}{
\fancyhead{}
\fancyhead[LE,RO]{
\sffamily\team
}
\fancyhead[RE,LO]{
\sffamily\leftmark
}
\renewcommand{\headrulewidth}{.4pt}
\cfoot{}
\fancyfoot[RO,LE]{\sffamily
  pag. \thepage{} di \pageref{LastPage}}
\fancyfoot[RE,LO]{\sffamily\doctitle}
\renewcommand{\footrulewidth}{.4pt}
}

% stile per gli indici
\fancypagestyle{toc}{
\fancyhead{}
\fancyhead[LE,RO]{
\sffamily\team
}
\fancyhead[RE,LO]{
\sffamily\caName
}
\renewcommand{\headrulewidth}{.4pt}
\cfoot{}
\fancyfoot[RO,LE]{\sffamily\thepage{}}
\fancyfoot[RE,LO]{\sffamily\doctitle{}}
\renewcommand{\footrulewidth}{.4pt}
}

\pagestyle{fancy}
\renewcommand{\sectionmark}[1]{\markboth{#1}{#1}}


%*******************************************************************************
% Inizio documento
%*******************************************************************************
\begin{document}

\pagestyle{empty}
\begin{center}

{\sffamily
Sviluppo e Gestione Progetti\\
a.a. 2012--2013
}

\vskip 1.5cm

\includegraphics[width=\textwidth]{logo}

\medskip
{\Huge\sffamily\bfseries
\team
}

\vskip 1.5cm

% titolo del progetto
{\Large\sffamily\bfseries
\caName
}

\vskip 1cm

% titolo del documento
\hrule
\vskip 10pt
{\Huge\scshape
\doctitle
}
\vskip 10pt
\hrule

\end{center}

\clearpage

\tableofcontents{\thispagestyle{toc}}

\clearpage

\pagestyle{normal}
\pagenumbering{arabic}

\chapter{Piano di Progetto}

%\section{Introduzione}

%\subsection{Scopo del documento}

% DESCRIZIONE DI ETA BETA
% \team si configura come una società di consulenza aziendale altamente specializzata in soluzioni basate sul paradigma BPM\@. Il \inglese{focus} della sua attività consiste nell'individuare e integrare le soluzioni di qualità esistenti -- sulla scorta di una profonda conoscenza delle alternative disponibili sul mercato sia nazionale che internazionale e con uno speciale interesse nei confronti delle innovazioni più \inglese{cutting edge} del momento -- integrandole con  soluzioni sviluppate \inglese{ad hoc} per i problemi particolari al fine di adattarsi con una precisione `sartoriale' alle esigenze e alle strategie dei singoli clienti.

% Le soluzioni proposte da \team rendono possibile innovare il proprio modello di \bsn monitorando in tempo reale lo svolgimento dei processi aziendali al fine di adottare misure correttive \inglese{in itinere} e, grazie alla consapevolezza maturata in tale sede, risolvere in maniera permanente le criticità nell'ottica del miglioramento continuo e del BPR (\inglese{Business Process Reengineering}).

% \team nasce nel 2010 con una missione fortemente orientata all'innovazione, allo scopo di fornire soluzioni \sw di alta qualità e a costi `ragionevoli' che mettano i propri clienti nelle condizioni di ottimizzare l'utilizzo delle risorse e migliorare in maniera sostenibile la propria efficienza e la propria competitività in un periodo particolarmente critico per l'economia come quello attuale.

% Il \inglese{team} di \team si compone di due esperti di \bsn \inglese{analysis} e organizzazione aziendale, uno specialista di \inglese{project management} e tre ingegneri del \sw come evidenziato nel diagramma riportato in \figurename~\ref{fig:organigram}.

% L'ambiente di collaborazione multidisciplinare che scaturisce da una simile combinazione di esperti con \inglese{background} afferenti a discipline profondamente differenti ma che nel corso del tempo hanno evidenziato importanti segnali di convergenza permette all'impresa di creare soluzioni apposite da integrare con quelle esistenti per conseguire gli obiettivi primari di pianificare, monitorare e automatizzare la gestione dei processi aziendali migliorandone l'efficienza ma senza sacrificare le spinte all'innovazione e mantenendo un basso profilo di costi.

% \team adotta internamente un \emph{modello integrativo} fortemente orientato al \inglese{commitment} che vede coinvolte al contempo lo sviluppo delle risorse interne, considerate come \inglese{asset} aziendali meritevoli di investimento in formazione continua piuttosto che come mere variabili di costo, e il controllo relativo alla \inglese{compliance} dei dipendenti rispetto agli standard di processo mirando alla costituzione di un ciclo virtuoso in cui le relazioni sociali e la fiducia reciproca sono incoraggiati nel rispetto degli obiettivi e senza perdere di vista il \inglese{focus} sui risultati aziendali.

% Le competenze maturate come analista di organizzazione aziendale e solida esperienza e nella gestione delle risorse umane del titolare consentono di mantenere sotto costante controllo l'operato dei dipendenti al fine di garantire il pieno rispetto della pianificazione e la qualità del risultato finale.

\section{Pianificazione}
Questa sezione espone la pianificazione prevista da \team per lo svolgimento delle attività di progetto. Viene qui esposta sia la pianificazione temporale che quella inerente alle risorse necessarie per portare a termine gli obiettivi di progetto.

\subsection{Work Breakdown Structure}

La \inglese{Work Breakdown Structure} (WBS) costituisce il primo passo verso una buona pianificazione.

La WBS sfrutta il concetto di `\textit{divide et impera}', essa infatti ha lo scopo di suddividere il lavoro di progetto in parti minori: in questo modo anche i progetti più complessi diventano realizzabili.

La scomposizione avviene secondo una struttura gerarchico: il lavoro viene dapprima suddiviso in una serie di attività di primo livello che vengono a loro volta scomposte in sotto-attività di secondo livello di dimensioni più piccole e semplici da gestire, dette \inglese{Work Package} (WP).

La rappresentazione della scomposizione avviene tramite l'uso di un diagramma ad albero, all'interno del quale i nodi interni rappresentano le attività e le foglie i singoli WP\@.

La WBS permette, tramite la suddivisione del lavoro in segmenti minori, di analizzare tutte le implicazioni del progetto senza tralasciare alcun aspetto. Inoltre, costituisce anche un buon punto di partenza per calcolare costi e benefici del progetto.

La WBS, infine, offre una chiara visione del prodotto finale e del processo complessivo attraverso il quale è stato creato il prodotto. Si presenta di seguito la definizione della WBS per il progetto BPM-1.0. 

Il \inglese{team} ha scelto di strutturare la WBS seguendo il ciclo di vita del progetto, individuando in primo luogo tre attività di primo livello che corrispondono alla Pianificazione, allo Studio di mercato e alla Proposta, come illustrato dal diagramma riportato in \figurename~\ref{fig:WBS}. 

Nelle sezioni successive sarà riportata in dettaglio la descrizione di ciascuna delle attività di primo livello e la descrizione relativa ad ogni WP in essa contenuto. 

\begin{figure}[H]
  \centering
  \includegraphics[width=\textwidth]{WBS}
  \caption{Work Breakdown Structure.}
  \label{fig:WBS}
\end{figure}

\subsubsection{Pianificazione}
In questa fase devono essere eseguite tutte le attività necessarie alla pianificazione del progetto.
Dovranno pertanto essere individuate ed organizzate tutte le attività da svolgere, i tempi e le risorse necessarie allo sviluppo del progetto. Tale fase comprende anche la stesura del \inglese{business plan}.
		
\paragraph{\bfseries\sffamily{}Analisi di mercato e Studio di fattibilità}
\begin{description}
  \item{\bfseries Descrizione:}\\
Questo WP prevede uno studio generale del mercato sui \sw BPM, con lo scopo di permettere l'organizzazione della pianificazione in base alle informazioni tratte dagli studi.
		
In particolare, lo studio di fattibilità dà al \inglese{team} la possibilità di capire se gli strumenti e le risorse di cui dispone sono sufficienti a garantire il conseguimento dell'obiettivo.

L'analisi, invece, permette di eseguire una corretta pianificazione sulla base di informazioni reali riguardanti il mercato attuale. 

Il risultato di tale WP è destinato a divenire la base per l'attività di Studio di mercato vera e propria e andrà in seguito a confluire all'interno del capitolo~\ref{sec:studiodimercato} del presente documento.
\item{\bfseries Attività:}
\begin{enumerate}
   \item ricerche relative ai \sw BPM;
   \item ricerche sui \inglese{competitor} del mercato;
   \item ricerche brevettuali.
\end{enumerate}
\item{\bfseries Costo:} \text{\euro} 350,00 
\item{\bfseries Tempo di realizzazione:} 2 giorni lavorativi
\end{description}

\paragraph{\bfseries\sffamily{}Pianificazione del lavoro}
\begin{description}
\item{\bfseries Descrizione:}\\
Questa attività prevede la pianificazione del lavoro in seguito allo studio di fattibilità. Il lavoro deve essere organizzato nelle sue attività e nei suoi tempi.
\item{\bfseries Attività:}
\begin{enumerate}
	\item stesura WBS;
	\item stesura OBS;
	\item matrice delle responsabilità;
	\item stesura RBS;
	\item realizzazione diagramma di Gantt.
\end{enumerate}
\item{\bfseries Costo:} \text{\euro} 600,00 
\item{\bfseries Tempo di realizzazione:} 2 giorni lavorativi
\end{description}

\paragraph{\bfseries\sffamily{}Redazione Business Plan}
\begin{description}
\item{\bfseries Descrizione:}\\
La \team si assume l'incarico di supportare la \customer nella redazione  del \inglese{business plan} del piano di sviluppo. Infatti, prevedendo il piano tra i suoi progetti, l'acquisto di \sw BPM, la \customer ha richiesto un intervento dell'azienda alla quale ha commissionato il compito della scelta del \sw.

Pertanto questo WP prevede la collaborazione di \team con la \customer per la redazione del \inglese{business plan}. Tale documento è molto importante sia a titolo organizzativo che rappresentativo, del progetto di sviluppo e dell'azienda stessa.
\item  {\bfseries Attività:}
\begin{enumerate}
	\item individuazione del contenuto del documento;
	\item individuazione informazioni necessarie alla stesura del documento;
	\item analisi del mercato e dei \inglese{competitor};
	\item studio dell'aspetto finanziario.
\end{enumerate}
\item{\bfseries Costo:} \text{\euro} 520,00 
\item{\bfseries Tempo di realizzazione:} 2 giorni lavorativi
\end{description}

\subsubsection{\bfseries\sffamily{}Studio di mercato}
In questa fase devono essere eseguite tutte le attività inerenti allo studio di mercato sui \sw BPM\@.

Il primo passo da compiere è l'analisi generica del mercato e l'individuazione dei prodotti esistenti. Sarà poi necessario selezionare alcuni \sw atti all'analisi approfondita in modo tale da poter scegliere la migliore soluzione. 

\paragraph{Studio dei \sw esistenti }
\begin{description}
\item{\bfseries Descrizione:}\\
Questo WP prevede l'analisi dei \sw già presenti nel mercato. Si tratta di analizzare le diverse soluzioni già esistenti in commercio ed individuare quelle più adatte all'obiettivo del progetto. 
\item{\bfseries Attività:}
\begin{enumerate}
	\item analisi generica delle soluzioni \sw già presenti nel mercato;
	\item selezione di una lista di \sw da visionare;
	\item analisi generale dei \sw.
\end{enumerate}
\item{\bfseries Costo:} \text{\euro} 965,00 
\item{\bfseries Tempo di realizzazione:} 6 giorni lavorativi
\end{description}

\paragraph{\bfseries\sffamily{}Studio dei \sw prescelti}
\begin{description}
\item{\bfseries Descrizione:}\\
Questo WP prevede l'analisi dei \sw selezionati. Si tratta di analizzare, studiare e confrontare le prestazioni dei vari \sw sulla base della lista delle funzionalità richieste da \customer.
\item  {\bfseries Attività:}
\begin{enumerate}
	\item individuazione della lista degli aspetti da tenere in considerazione;
	\item analisi individuale di ogni \sw;
	\item confronto dei diversi \sw.
	\end{enumerate}
\item{\bfseries Costo:} \text{\euro} 1.100,00 
\item{\bfseries Tempo di realizzazione:} 7 giorni lavorativi
\end{description}

\paragraph{\bfseries\sffamily{}Analisi Proposta nuovo \sw o soluzione ibrida }
\begin{description}
\item{\bfseries Descrizione:}\\
Questo WP prevede l'analisi della possibilità della realizzazione di un nuovo \sw o di una soluzione ibrida.

Tale analisi avverrà soltanto dopo aver esaminato i diversi \sw selezionati e la scelta di sviluppare una soluzione \inglese{ad hoc} sarà scaturita dalla possibilità che nessun \sw attualmente in commercio soddisfi pienamente le richieste di \customer.

\newpage

\item{\bfseries Attività:}
\begin{enumerate}
  \item valutazione generale del rispetto dei requisiti posti da \customer;
	\item valutazione di costi e benefici della creazione di una soluzione \inglese{ad hoc} \inglese{software}.
\end{enumerate}
\item{\bfseries Costo:} \text{\euro} 335,00 
\item{\bfseries Tempo di realizzazione:} 2 giorni lavorativi
\end{description}

\subsubsection{Proposta}
In questa fase devono essere eseguite tutte le attività inerenti alla stesura delle proposta. Deve essere scelto il prodotto da proporre e individuato il modo per farlo.

Si prevede inoltre di realizzare un piccolo manuale per l'uso del prodotto \sw.

\paragraph{\bfseries\sffamily{}Scelta \sw}
\begin{description}
\item{\bfseries Descrizione:}\\
Questo WP rappresenta un punto cruciale. Dalla scelta del \sw dipende il contenuto della proposta che sarà presentata.

Si tratta quindi di decidere  nel migliore dei modi quel è il \sw BPM più adatto alle esigenze richieste di \customer.
\item{\bfseries Attività:}
\begin{enumerate}
	\item valutazione rispetto ai requisiti obbligatori concordati con \customer;
	\item valutazione rispetto ai requisiti desiderabili concordati con \customer;
	\item decisione definitiva con i responsabili.
\end{enumerate}
\item{\bfseries Costo:} \text{\euro} 345,00 
\item{\bfseries Tempo di realizzazione:} 2 giorni lavorativi
\end{description}

\paragraph{\bfseries\sffamily{}Redazione della proposta }
\begin{description}
\item{\bfseries Descrizione:}\\
Una volta che la scelta del \sw è avvenuta, sarà necessario presentarla in modo formale all'azienda richiedente.

Infatti, la forma con cui viene presentato il lavoro svolto è fondamentale per determinare il successo del progetto e portare a termine gli obiettivi prefissati.
\item  {\bfseries Attività:}
\begin{enumerate}
	\item individuazione del contenuto della proposta;
	\item stesura del documento;
	\item verifica del documento;
 	\item approvazione del documento.	
\end{enumerate}
\item{\bfseries Costo:} \text{\euro} 420,00 
\item{\bfseries Tempo di realizzazione:} 2 giorni lavorativi
\end{description}

\paragraph{\bfseries\sffamily{}Redazione manuale/Consigli d'uso}
\begin{description}
\item{\bfseries Descrizione:}\\
Questo WP consiste nella stesura di un breve manuale per l'uso del \inglese{software} con il fine di facilitare gli utenti che ne faranno uso.
\item{\bfseries Attività:}
\begin{enumerate}
	\item individuazione del contenuto del manuale
	\item scelta della forma di presentazione
	\item verifica 
 	\item approvazione	
\end{enumerate}
\item{\bfseries Costo:} \text{\euro} 60,00 
\item{\bfseries Tempo di realizzazione:}  1 giorno lavorativo
\end{description}

\paragraph{\bfseries\sffamily{}Presentazione proposta}
\begin{description}
\item{\bfseries Descrizione:}\\
Questo WP costituisce la conclusione di tutto il lavoro di progetto svolto. Si tratta di 		riassumere in una breve presentazione i contenuti del progetto e in particolari i motivi che hanno portato alla scelta di un determinato \inglese{software} BPM.
% 	\item {\bfseries Responsabile:}
\item{\bfseries Attività:}
\begin{enumerate}
	\item individuazione del contenuto della presentazione
	\item creazione della presentazione
\end{enumerate}
\item{\bfseries Costo:} \text{\euro} 80,00 
\item{\bfseries Tempo di realizzazione:}  1 giorno lavorativo
\end{description}

\subsection{Organizational Breakdown Structure}
L'\inglese{Organizational Breakdown Structure} (OBS) rappresenta l'organizzazione del progetto rispetto alle risorse umane impiegate in esso.
L'OBS rappresenta una scomposizione gerarchica delle responsabilità di progetto, generata alla scopo di individuare i responsabili di ogni WP.

L'OBS deve essere creata in seguito alla redazione della WBS\@. Infatti, solo dopo aver tracciato i WP è possibile assegnare loro un responsabile/esecutore.

La creazione della OBS risulta utile sia sotto l'aspetto gerarchico in quanto permette al \inglese{Project Manager} di individuare i responsabili, sia sotto l'aspetto organizzativo degli esecutori materiali delle attività, in quanto facilita la comunicazione tra essi permettendo di capire a chi chiedere cosa. 

\begin{figure}[h!]
  \includegraphics[width=\textwidth]{OBS}
	\caption{Organizational Breakdown Structure.}
	\label{fig:obs}
\end{figure}

Come si evince dal diagramma riportato in \figurename~\ref{fig:obs}, tutti i responsabili sono a capo del titolare dell'azienda. Essendo infatti una piccola azienda il controllo, pur essendo distribuito rimane comunque sotto la supervisione del titolare.
Il \inglese{Project Manager} è invece a capo dell'Analista e del Programmatore. 
 
Si precisa, infine, che le risorse umane impiegate nel progetto sono tutte e sole quelle di cui dispone l'azienda \team.

\subsubsection{Titolare}
Il Titolare è colui che possiede l'azienda (in tutto o in parte).
	
\subsubsection{Amministratore}		
L'amministratore è una figura molto importante dal punto di vista organizzativo. Ogni decisione deve essere approvata da 	tale figura e perciò deve essere messo al corrente di cosa succede sia nelle attività ordinarie che in quelle di progetto.
	
	Egli è responsabile dell'efficienza e dell'operatività dell'ambiente di sviluppo. Controlla inoltre versioni e configurazioni del prodotto.
	
\subsubsection{Project Manager}
Il \inglese{Project Manager} ricopre un ruolo determinante nella gestione dei progetti. Questi è infatti responsabile della valutazione, pianificazione, realizzazione e controllo di un progetto.
	
I suoi compiti più importanti sono:
\begin{itemize}
	\item valutazione di costi e benefici del progetto
  \item pianificazione del progetto
	\item pianificazione e gestione dei rischi di progetto
	\item valutazione dello stato di avanzamento del progetto
	\item adozione di misure correttive qualora l'avanzamento del progetto non corrispondesse alla pianificazione
\end{itemize}

\subsubsection{Responsabile Marketing}
Questa figura si occupa della definizione ed applicazione delle strategie di \mktg. Ad essa spetta anche il ruolo di dedicarsi alla gestione delle relazioni con i clienti e della promozione delle attività dell'azienda stessa.
	
Tale figura si è incaricata inoltre, del monitoraggio dei dati di mercato, degli indicatori e dei canali di vendita, al fine di permettere l'identificazione di nuove opportunità di \bsn. La persona che ricopre il tale ruolo deve essere in possesso di ottime capacità relazionali, manageriali e di \inglese{leadership}. 
	
\subsubsection{Responsabile Qualità}	
La qualità è fondamentale per avere successo nei progetti. Oggi è una proprietà irrinunciabile, sia per clienti che fornitori. Per i clienti si tratta infatti di un'assicurazione sul prodotto/servizio che acquistano e per i fornitori, invece, costituisce  un punto di distinzione rispetto ai concorrenti.
	
Pur essendo una piccola azienda, \team punta al raggiungimento della qualità, tramite l'adeguamento allo \inglese{standard} ISO/IEC 9001.
Nell'ambito dei progetti aziendali il Responsabile Qualità deve assicurare l'attuazione dei processi volti a garantire il rilascio di un prodotto/servizio che rispetta tutti i requisiti di qualità stabiliti.
	 
\subsubsection{Analista}
L'Analista ha un ruolo fondamentale nella fase iniziale del progetto. Tale figura deve infatti effettuare lo studio di fattibilità preoccupandosi dell'aspetto tecnico.

Nel caso di specie l'Analista ricopre un ruolo molto importante anche nella fase di sviluppo. Egli sarà, infatti, una delle persone addette alla valutazione dei \sw esistenti nel mercato e perciò  dalla sua stima dipende in gran parte il successo o il fallimento del progetto.

\subsubsection{Programmatore}
	Tale ruolo, in generale, dovrebbe limitarsi alla sola attività di codifica seguendo le direttive progettuali per sviluppare il prodotto finale. In questo progetto, essendo l'azienda molto piccola, egli avrà il compito di supportare l'Analista nello studio di fattibilità e di collaborare con quest'ultimo all'analisi dei \sw avvalendosi della propria personale esperienza.

\subsection{Matrice delle responsabilità}
La rappresentazione delle assegnazioni delle responsabilità tramite una matrice permette la definizione del flusso di comunicazioni all'interno della struttura organizzativa del progetto.

La matrice, permette infatti, di individuare all'interno dell'organizzazione non solo i responsabili di una certa attività ma anche quei soggetti che devono essere consultati o informati.

La \figurename~\vref{fig:WBS_OBS} illustra come la matrice delle responsabilità rappresenta l'incrocio tra la WBS e l'OBS\@. Il diagramma chiarisce infatti chi fa cosa.

La \figurename~\vref{fig:MATR} rappresenta invece la matrice delle responsabilità in modo da individuare anche la specifica responsabilità di ogni figura. Si noti che tale matrice contiene attività molto specifiche, come l'Approvazione, che non si è ritenuto importante inserire nel WBS.

Inoltre si fa notare che per attività di `Pianificazione Generale' si intende una pianificazione poco precisa che ha il solo fine di capire se l'azienda poteva candidarsi per il capitolato. Infine, per quanto riguarda l'attività di `Preparazione Offerta' si intende la candidatura stessa del \inglese{team} mentre la `Firma del Contratto' si riferisce all'approvazione da parte del docente.


\begin{figure}[h]
  \includegraphics[width=1.3\textwidth]{WBS_OBS}
	\caption{Matrice delle responsabilità e WBS.}
	\label{fig:WBS_OBS}
\end{figure} 

\begin{figure}[p]
  \includegraphics[width=0.9\textwidth]{matrice}
	\caption{Matrice delle Responsabilità.}
  \label{fig:MATR}
\end{figure}

\subsection{Resource Breakdown Structure}
Una volta definite quali sono le attività da svolgere e chi ne è responsabile è buona norma effettuare la pianificazione di tutte le risorse necessarie alla svolgimento del progetto, sia umane che strumentali.

Tale organizzazione viene effettuata con lo strumento di pianificazione \inglese{Resource Breakdown Structure} (RBS).
Lo scopo principale di RBS è esplicitare tutte le risorse necessarie e le relazioni che intercorrono tra esse, organizzando queste informazioni in un diagramma gerarchico ad albero e classificandole per categoria.

La \figurename~\vref{fig:RBS} illustra la classificazione delle risorse necessarie allo svolgimento del presente progetto.

\begin{landscape}
\vskip 2in

\begin{figure}[p]
\centering
  \includegraphics[width=1.5\textwidth]{RBS}
	\caption{Resource Brakedown Structure.}
	\label{fig:RBS}
\end{figure}

\end{landscape}

Come si evince dal diagramma riportato in \figurename~\ref{fig:RBS}, le risorse necessarie sono state suddivise gerarchicamente in interne ed esterne, e a loro volta in umane e strumentali. Tale suddivisione permettere di capire dove collocare ogni risorsa necessaria al progetto.

Si noti che vengono considerate esclusivamente le risorse che comportano un costo di cui si dovrà rientrare con gli utili generati dal progetto. Di conseguenza non sono menzionate risorse come \sw con licenza gratuita.

\subsection{Pianificazione Temporale}
Per la pianificazione temporale si utilizza il diagramma di Gantt. Tale diagramma rappresenta infatti l'evoluzione del progetto su scala temporale.

Il Gantt costituisce non solo un ottimo strumento di pianificazione ma anche un buon strumento di controllo che permette in ogni momento di verificare lo stato di avanzamento del progetto rispetto a tempi ed attività pianificate.

Esso è costituito da un asse orizzontale, sul quel viene rappresentato l'arco temporale totale del progetto e un asse verticale, sul quale sono rappresentate tutte le attività ed i singolo WP che costituiscono il progetto.

La rappresentazione della distribuzione sull'asse temporale delle attività di progetto è fornita in \figurename~\ref{fig:gantt}.

\begin{figure}[H]
  \includegraphics[width=\textwidth]{gantt}	
	\caption{Diagramma di Gantt.}
	\label{fig:gantt}
\end{figure}

\section{Aspetto Economico Finanziario}\label{sec:aspettoeconomico}
Il lato economico costituisce un aspetto di interesse primario di ogni organizzazione che abbia scopo di lucro. È infatti fondamentale avere un utile sufficientemente remunerativo alla fine del progetto.

A tale scopo è finalizzata la stima dei costi relativi al progetto. Tale attività è molto delicata in quanto da essa dipende il reale utile che si ottiene. Infatti, se vi è una sottostima dei costi, l'azienda non avrà utile e nel caso peggiore avrà una perdita.

È anche vero, però, che se vi è una sovrastima dei costi, l'azienda non potrà essere competitiva nel mercato. Il processo di stima avviene calcolando il costo per ogni WP individuato durante il processo di creazione del WBS.

\subsection{Ruoli e Costo}
La seguente tabella espone i costi orari per ogni ruolo individuato. Si evidenzia, in particolare, che tutte le risorse sono interne ad eccezione del consulente.

\begin{table}[H]
\centering
\begin{tabular}{|p{.25\textwidth}|c|c|}
\hline
\textbf{Ruolo}& \textbf{Costo Medio} \\
\hline
Amministratore			  & \EUR{40,00}\\
\hline
Project Manager			  & \EUR{30,00}\\
\hline
Responsabile MKTG		  & \EUR{25,00}\\
\hline
Responsabile Qualità	& \EUR{20,00}\\
\hline
Analista				      & \EUR{25,00}\\
\hline
Programmatore		     	& \EUR{15,00}\\
\hline
Consulente				    & \EUR{30,00}\\
\hline
\end{tabular}
\caption{Costo medio orario dei ruoli aziendali.}
\label{tab:pianificazione}
\end{table} 

\subsection{Costi per Attività}

\subsubsection{Pianificazione}
Per l'attività di Pianificazione il costo previsto è di \textbf{\EUR{1.470,00}}\\	
I costi per ogni WP della Pianificazione sono riassunti nella seguente tabella.
\begin{table}[H]
\footnotesize
\centering
\begin{tabular}{|p{.25\textwidth}|c|c|c|c|c|c|r|}
\hline
\textbf{Attività}& \textbf{AM} & \textbf{RMKTG} & \textbf{PM} & \textbf{RQ} & \textbf{PRG} & \textbf{AN} & \textbf{Costo}  \\
\hline
analisi del problema e studio di fattibilità  & & & 5& & & 8& \EUR{350,00}\\
pianificazione del lavoro	 				            & & &	16&	6& & & \EUR{600,00}\\	
redazione \inglese{business plan}	            & 1 & &16& & & &  	\EUR{520,00}\\			  
\hline
\scshape{}pianificazione   							      & 1 & &37 &	6 &	&	8 &	\textcolor{red}{\EUR{1.470,00}}\\		 
\hline
\end{tabular}
\caption{costo attività `Pianificazione'}\label{tab:pianificazione}
\end{table}

\subsubsection{Studio di Mercato}

Per l'attività di Studio di Mercato il costo previsto è di \textbf{\EUR{2.300,00}}\\	
I costi per ogni WP dello Studio di Mercato sono riassunti nella seguente tabella.

\begin{table}[H]
\footnotesize
\centering
\begin{tabular}{|p{.25\textwidth}|c|c|c|c|c|c|r|}
\hline
\textbf{Attività}& \textbf{AM} & \textbf{RMKTG} & \textbf{PM} & \textbf{RQ} & \textbf{PRG} & \textbf{AN} & \textbf{Costo}  \\ 
            
\hline
studio software esistenti & & & 1& & 24& 23& \EUR{965,00}\\
studio software prescelti	 				  & & &	&	&40 &16 & \EUR{1.000,00}\\	
analisi proposta nuovo SW/soluzione ibrida 					  & & &2 & 	&5	&  8  & \EUR{335,00}\\			  
\hline
\scshape{}studio di mercato  							& 1 &  &3 & &69	&47	&	\textcolor{red}{\EUR{2.300,00}}\\		 
\hline
\end{tabular}
\caption{costo attività `Studio di Mercato'}\label{tab:mercato}
\end{table}

\subsubsection{Proposta}
Per l'attività di Redazione della Proposta il costo previsto è di \textbf	{\EUR{905,00}}	
I costi per ogni WP della Redazione della Proposta sono riassunti nella seguente tabella.

\begin{table}[H]
\footnotesize
\centering
\begin{tabular}{|p{.25\textwidth}|c|c|c|c|c|c|r|}
\hline
\textbf{Attività}& \textbf{AM} & \textbf{RMKTG} & \textbf{PM} & \textbf{RQ} & \textbf{PRG} & \textbf{AN} & \textbf{Costo}  \\ 
\hline
scelta software			& & & 3&	& 7&	6& \EUR{345,00}\\
redazione proposta & 1&	8&	6& & & & \EUR{420,00}\\
redazione manuale / consigli d'uso & & & & & 					4 && \EUR{60,00}\\	
presentazione proposta		 & & 2&  	1	& & 	& & \EUR{80,00}\\	
\hline
\scshape{}proposta  							& 1  &10 &10& &	4&	6&	\textcolor{red}{\EUR{905,00}}\\		 
\hline
\end{tabular}
\caption{costo attività `Redazione Proposta'}\label{tab:proposta}
\end{table}	
	
	
\subsubsection*{Consulenza}

Si noti che nelle tabelle relative alle attività non sono inclusi i costi inerenti alla figura del consulente. Il \inglese{team} ha infatti deciso di trattarli a parte. La seguente tabella indica i costi che \team prevede di sostenere per le spese di consulenza in ogni fase.
	
\begin{table}[H]
\centering
\begin{tabular}{|l|c|r|}
\hline
\textbf{Attività}& \textbf{Consulente} & \textbf{Costo}  \\           
\hline
\scshape{}pianificazione		& 1 & \EUR{30,00}\\
\scshape{}studio di mercato & 5 & \EUR{150,00}\\
\scshape{}proposta 			    &   &	\EUR{60,00}\\	
\hline
Totale				              & 6 & \EUR{240,00}\\	
\hline
\end{tabular}
\caption{costo attività `Consulenza'}\label{tab:consulenza}
\end{table}

Come si evince dalla tabella~\ref{tab:consulenza} il costo stimato per le spese di consulenza ammonta a \textbf{\EUR{240,00}}.

\subsection{Costo Totale}

Il costo totale stimato ammonta quindi a \textbf{\EUR{4.915,00}}, tale cifra è da considerarsi come comprensiva di un utile sufficientemente remunerativo ma che allo stesso tempo permette a \team di essere competitiva nel mercato. I costi stimati costituiscono un preventivo, \team si riserva di apportare eventuali variazioni a consuntivo.

\section{Gestione dei Rischi}
Un rischio è un evento incerto o una condizione di incertezza che se accade determina un effetto positivo o negativo sull'obiettivo di progetto. La gestione dei rischi è fondamentale in un progetto. Infatti, da una gestione corretta o errata del rischio dipende il successo o il fallimento del progetto.

È quindi necessario individuare quali rischi possono verificarsi durante il progetto e pianificare tecniche e strategie per evitare o, nel caso peggiore, mitigare tali rischi.

La valutazione dei rischi non deve essere un processo statico ma dinamico. Non è infatti sufficiente individuare rischi e tattiche all'inizio del progetto ma è necessario effettuare azioni di controllo costanti e rivedere le strategie correttive qualora non si rivelassero adatte. 

\subsection{Analisi dei rischi}

\newcommand{\hi}{\textsc{alta}}
\newcommand{\lo}{\textsc{bassa}}
\newcommand{\med}{\textsc{media}}
Per rendere efficace l'analisi di ogni rischio si è deciso di quantificarlo mediante un apposita scala di valutazione sia dal punto di vista della probabilità che il rischio si manifesti (livello), sia il suo grado di incidenza sul progetto stesso (impatto).

\begin{table}[h!]
\centering
\begin{tabular}{|l|c|}
\hline
\bfseries{}Probabilità& \bfseries{}Descrizione\\
\hline
\hi & probabilità elevata che si verifichi\\
\med & probabilità equivalente nel verificarsi o meno\\
\lo & probabilità bassa che si verifichi\\
\hline
\end{tabular}
\caption{Probabilità e Descrizione probabilità di un rischio}\label{tab:livellorischi}
\end{table}
\begin{table}[H]
\centering
\begin{tabular}{|c|c|}
\hline
\bfseries{}Scala& \bfseries{}Descrizione  \\
\hline
5 & conseguenze molto gravi\\
4 & conseguenze gravi\\
3 & conseguenze medio-gravi\\
2 & conseguenze minimali\\
1 & nessuna/lievi conseguenze\\
\hline
\end{tabular}
\caption{Scala e descrizione delle conseguenze di un rischio}\label{tab:impattorischi}
\end{table}

\subsection{Rischi relativi al Personale}
\begin{description}
	\item{\scshape\bfseries Analisi:}\\
	Durante la realizzazione del progetto è probabile che alcuni membri del team siano soggetti a problemi fisiologici e/o sovvengano impegni personali improrogabili che porterebbero ad una sicura modifica della pianificazione del lavoro collettivo.
	
	L'impatto di tale rischio è variabile in base al soggetto mancante, in quanto può essere assegnato ad un'attività più o meno importante all'interno del progetto. 
	\item{\scshape\bfseries Probabilità:} \med
	\item{\scshape\bfseries Impatto:} variabile
	\item{\scshape\bfseries Strategia di Gestione:}\\
	Per mitigare gli effetti di tali fenomeni è ragionevole prima di tutto pianificare i tempi di lavoro personali in modo da lasciare un lasco temporale tra un attività e l'altra.
	
Purtroppo, data la scarsità di tempo a disposizione, si dispone in questo caso di poco tempo margine. Tuttavia, essendo il periodo che impegnerà il \inglese{team} breve, vi è una cospicua probabilità che le attività procedano come pianificate.
	
Ovviamente anche adottando tali accorgimenti si potrà generare la situazione in cui un componente risulti impossibilitato a svolgere il proprio compito, in tal caso è buona norma che tutti i membri siano ben preparati (conoscenza del dominio e delle metodologie di lavoro) nel caso sia necessaria la sostituzione momentanea del soggetto.
\end{description}

\subsection{Rischi relativi alla tecnologia}
\begin{description}
	\item{\scshape\bfseries Analisi:}\\
	Per ovvie ragioni di inesperienza da parte di tutto il team buona parte delle competenze tecnologiche richieste per la realizzazione del progetto risultano sconosciute.
	\item{\scshape\bfseries Probabilità:} \hi 
	\item{\scshape\bfseries Impatto:} 3 
	\item{\scshape\bfseries Strategia di Gestione:}\\
	Le lacune saranno colmate tramite la personale consultazione di materiale presente in rete. Inoltre, ogni persona provvederà ad aggiornare costantemente gli altri membri del \inglese{team} sul lavoro svolto.
\end{description}

\subsection{Rischi relativi all'errata stima di risorse}
\begin{description}
	\item{\scshape\bfseries Analisi:}\\
	L'errata pianificazione del lavoro fa parte dell'ovvia inesperienza del team, e in particolare di chi ricopre il ruolo di Project Manager. Tali errori possono portare ad uno sbilanciamento dei costi (sia in eccesso che in difetto) che andrà ad incidere sul bilancio finale.
	\item{\scshape\bfseries Probabilità:} \med
	\item{\scshape\bfseries Impatto:} 3 
	\item{\scshape\bfseries Strategia di Gestione:}\\
	Per mitigare gli effetti di tali rischi il \inglese{team} ha cercato di fare ricerche molto approfondite sugli argomenti e ha effettuato una pianificazione leggermente ``pessimistica''. In questo modo viene ridotta la probabilità di incorrere ad eventuali perdite che minerebbero la solidità economico-finanziaria dell'azienda.
\end{description}

\subsection{Rischi relativi al mercato}
\begin{description}
	\item{\scshape\bfseries Analisi:}\\
	Il \inglese{team} non ha nessuna esperienza sulla situazione di mercato dei \inglese{software} BPM\@. Tale carenza, purtroppo, può incidere molto sulla tempistica di realizzazione del progetto.
	\item{\scshape\bfseries Probabilità:} \hi
	\item{\scshape\bfseries Impatto:} 3 
	\item{\scshape\bfseries Strategia di Gestione:}\\
	Per mitigare gli effetti di tali rischi il \inglese{team} potrà avvalersi di eventuali consulenti e dell'aiuto del docente.
\end{description}	

%*******************************************************************************
% STUDIO DI MERCATO
%*******************************************************************************

\chapter{Studio di Mercato}\label{sec:studiodimercato}
\section{Introduzione}
\subsection{Cosa sono i software BPM}
	
Tutte le aziende lavorano per processi: essi sono il cuore e l'anima di ogni organizzazione.
		
Uno dei punti che determinano il successo di un'azienda è costituito dall'organizzazione e gestione dei processi stessi. I \sw BPM permettono di gestire i processi facendo uso della tecnologia. 
		
Si tratta di \sw all'avanguardia che permettono la pianificazione, la gestione ed il monitoraggio dei processi.
I \sw BPM permettono  ai \inglese{leader} di grandi e piccole aziende, di avere una migliore comprensione, di prendere decisioni più rapide e cosa più importante, di eliminare il caos e l'inefficienza che possono avere un'incidenza sul vantaggio competitivo di un'azienda.

Spesso questi \sw permettono di interagire con altri \sw gestionali, database o PIM\@.\footnote{I \inglese{Personal Information Manager} sono \sw che permettono di organizzare un certo tipo di informazioni personali per migliorare la produttività aziendale. In genere questi \sw sono usati per gestire \inglese{email} e rubriche.}

Infine, i \sw BPM permettono una corretta gestione dei flussi informativi. Infatti, almeno il 70\% delle informazioni non sono sufficientemente strutturate e quindi perdono valore, come emerge da alcuni studi come ad esempio \cite{mazzolari:bpmita}. Grazie ai \sw BPM è possibile massimizzare l'apporto di valore fornito dalle informazioni ed essere quindi competitivi nel mercato.

\subsection{Vantaggi}
L'adozione di una soluzione di BPM ha il vantaggio di comportare un incremento dell'efficienza a seguito dell'automazione dei processi nonché l'eliminazione degli \inglese{step} intermedi non necessari dovuta all'acquisizione di maggior consapevolezza derivante dalla formalizzazione dei processi aziendali in modelli astratti (situazione corrente `as is').

Implicando la standardizzazione del metodo di lavoro, l'implementazione di un \inglese{workflow management system} conseguenza implicita dell'adozione di un BPM permette di utilizzare di strumenti di verifica e di generazione automatica di \inglese{report} sulle attività ordinarie dell'azienda che consentono di monitorare gli indicatori chiave con estrema accuratezza. Si tratta di condizioni imprescindibili per una buona pianificazione e per l'ottimizzazione dei processi (situazione ideale o `to be') e per passare dalla semplice modellazione dei processi aziendali (\bsn \inglese{process modeling}) a una rivisitazione critica degli stessi (\bsn \inglese{process reengineering}).

In altre parole, i \sw BPM permettendo un'attenta organizzazione ed un continuo controllo dei processi, consentono di ottimizzare la produttività dell'azienda e di ottenere un alto livello di qualità. Inoltre, disponendo delle funzionalità di automatizzazione di alcuni processi, si acquisisce velocità e si rende così l'azienda molto più competitiva nel mercato.

La gestione della non-linearità dei processi, inoltre, denota la flessibilità di tali sistemi, che sono in grado di soddisfare le istanze di modellazione di situazioni che non sono prevedibili aprioristicamente, per adattarsi ai cicli di vita di sempre più ridotte dimensioni per la gestione degli ordini e a un ambiente competitivo in costante evoluzione.

Infine, tramite una gestione controllata dei processi, è possibile incrementare l'efficienza anche in caso di \inglese{workflow} collaborativi, rendendo più trasparente lo scambio di informazioni fra la totalità dei soggetti coinvolti, la condivisione della conoscenza nonché la drastica riduzione dei tempi di accesso alle basi di conoscenza aziendali, incrementandone al contempo l'utilità e le dimensioni perché il flusso di lavoro diviene tracciato, codificato e ripetibile anche in caso di situazioni potenzialmente complesse in maniera indipendente dalla soggettività e dall'esperienza degli incaricati.

\begin{figure}[H]
  \centering
  \includegraphics[width=.6\textwidth]{bpmcycle}
  \caption{Il tipico ciclo del BPM nelle attività aziendali.}
  \label{fig:bpmcycle}
\end{figure}

In sostanza, come esemplificato dalla \figurename~\ref{fig:bpmcycle}, tramite l'adozione di una soluzione BPM si instaura all'interno dell'organizzazione in circolo virtuoso in cui i processi di \bsn vengono modellati, eseguiti secondo la pianificazione, monitorati, corretti e, se necessario, ottimizzati al fine di produrre una migliore progettazione da cui si può trarre un nuovo modello nell'ottica del miglioramento continuo.

Possiamo quindi riassumere i vantaggi che scaturiscono dall'adozione di \sw BPM nel seguente elenco:

\begin{itemize}
	\item miglioramento dell'efficienza dei processi
	\item miglioramento del flusso informativo
	\item controllo coordinato e continuativo dello stato dei processi
	\item miglioramento nella pianificazione del lavoro
	\item attivazione più veloce di interventi correttivi
	\item maggior flessibilità verso situazioni non prevedibili
	\item dematerializzazione dei documenti 	
\end{itemize}
	
\subsection{Cosa offre il mercato}\label{sec:currentmarket}
Attualmente il mercato offre diverse soluzioni, sia gratuite che a pagamento, più, o meno, complete.	
Dall'analisi di mercato effettuata, emerge che in generale, le piccole aziende si affidano alle soluzioni gratuite, alcune delle quali sembrano coprire la maggior parte dei requisti richiesti da tali aziende. Le organizzazioni più complesse, invece, preferiscono acquisire \sw non gratuiti perché a volte risultano essere più completi ed adatti ad organizzazioni più grandi.

Occorre tuttavia tenere presente che nel mercato vi sono \sw distribuiti con una licenza \inglese{open source} e gratuiti, come \swname{Bonita BPM}, che sono in grado di gestire situazioni molto complesse e offrono numerose funzionalità anche di alto livello come parte del pacchetto di base per cui non è necessario sostenere alcun costo di acquisizione.
 
I \inglese{leader} attuali del mercato mondiale, in particolare quello legato alle soluzioni \sw a pagamento, sono grandi colossi del calibro di \swname{IBM}, \swname{Microsoft} e \swname{Oracle}. Tali organizzazioni hanno compreso fin da subito la necessità di offrire alle aziende \sw per la gestione di processi e si sono affermate nel tempo offrendo soluzioni con funzionalità ad alto livello che permettono anche l'interazione con \sw di terze parti.\footnote{%
Un esempio dell'interoperabilità tra le soluzioni offerte e applicazioni esterne è rappresentato da \swname{Bonita Open Solution} che permette, sottoscrivendo un abbonamento di tipo `\textsf{efficiency}', l'interfacciamento con applicativi \swname{SAP}.
}

Il seguente elenco riporta i \inglese{leader} del settore in ordine di rilevanza:
\begin{enumerate}
  \item \swname{IBM}
  \item \swname{Oracle}
  \item \swname{Microsoft}
\end{enumerate}

\swname{IBM} si configura come \inglese{leader} mondiale in vari segmenti, tra cui \inglese{Database Management}, \inglese{Enterprise Content Management} (ECM), \inglese{Customer Data Integration}, \inglese{Information Integration}, \inglese{Information On Demand} e, in generale, Service Oriented Architecture (SOA), operando in 173 paesi ed è presente in Italia dal 1927 svolgendo anche attività di sviluppo software, che fanno capo al \inglese{Rome Tivoli Laboratory}.

Nel campo della \bsn \inglese{intelligence} sono state particolarmente significative le acquisizioni di \swname{Lotus} nel 1995, \swname{Lombardi} nel 2005, \swname{MRO} (marchio \swname{Tivoli}) e \swname{FileNet} nel 2006 nonché \swname{Cognos} nel 2008. Il \inglese{portfolio} soluzioni della divisione \swname{IBM Software Group} comprende allo stato attuale \inglese{brand} fra i quali si menzionano \swname{Information Management}, \swname{Lotus}, \swname{Rational}, \swname{Tivoli}, \swname{WebSphere} e \swname{Product Lifecycle Management}.

\swname{Microsoft} si colloca nel mercato con una soluzione di gestione di database (\swname{Microsoft SQL Server}), collaboraborazione e condivisione di contenuti aziendali (\swname{Microsoft SharePoint}) e la \inglese{suite} di produttività individuale \swname{Microsoft Office}.

\swname{Oracle} si colloca nel mercato applicazioni di \bsn \inglese{intelligence} con prodotti propri e ha rafforzato ulteriormente la propria posizione di dominanza con l'acquisizione di \swname{Hyperion} che risale al 2007. Il risultato è una combinazione completa di strumenti di pianificazione, modellazione delle opzioni strategiche, allineamento organizzativo e reportistica strettamente integrati e funzionali. Attualmente, oltre 12.000 aziende in 91 paesi si affidano a soluzioni \swname{Oracle} per migliorare la comprensione del proprio \bsn e per la realizzazione di iniziative di \inglese{Enterprise Performance Management} (EPM).

Fra le altre società presenti sul mercato, vale la pena di menzionare \swname{SAP}, \swname{Fujistu} e \swname{HP}.

\swname{SAP} è presente con soluzioni di \inglese{data warehousing} come \swname{SAP HANA} e \swname{Sybase IQ} e \sw di monitoraggio e notifica dei processi aziendali (\swname{SAP BusinessObjects Event Insight}) attraverso il controllo degli indicatori di processo  e accordi sul livello di servizio (di tipo prestazionale).

\swname{Fujitsu}, dal canto suo, fornisce soluzioni per la gestione di grandi moli di dati aziendali attraverso la \inglese{suite} \swname{EternusSF} e per l'organizzzione e l'allineamento di risorse informatiche e \inglese{cloud-based} con i prodotti della linea \swname{ServerView}.

\swname{HP} partecipa con prodotti come \swname{HP Business Decision Applicance} e \swname{HP Data Warehouse Appliance} orientati principalmente alle gestione delle informazioni aziendali e all'\inglese{data mining} al fine di facilitare le attività di \inglese{decision making}, integrandosi strettamente con i prodotti \swname{Microsoft} più diffusi per la memorizzazione delle informazioni, come \swname{Microsoft SQL Server}.

Dagli studi di mercato emerge che le maggiori problematiche in questo settore riguardano la difficoltà delle piccole piccole/medie imprese a stare al passo con un mercato sempre più dinamico e la necessità di applicare la cosiddetta `modalità a latenza zero'.\footnote{Con tale espressione si intende l'accesso e la visibilità dello stato di avanzamento dei processi in tempo reale.}

Inoltre si stanno attualmente aprendo della opportunità anche per il settore della Pubblica Amministrazione che si pone l'obiettivo di offrire servizi di qualità.

Infine, molti dei \sw attualmente presenti nel mercato integrano un sistema di gestione documentale in conformità al formato sostitutivo previsto per legge.

Come si evince dal grafico riportato in \figurename~\ref{fig:tipologie}, la quota maggioritaria delle aziende che fanno utilizzo di \sw BPM è detenuta dalle società di consulenza come \customer che ha richiesto a \team di valutare una soluzione che la supporti nello svolgimento delle proprie attività. A seguire, si collocano le \sw \inglese{house} con una quota del 25\%, mentre la categoria dei \inglese{System Integrator} si attesta al 10\%.

\begin{figure}[H]
  \centering
  \includegraphics[width=.85\textwidth]{tipologie}
  \caption{Tipologie di aziende che investono in \sw BPM.}
  \label{fig:tipologie}
\end{figure}

\subsection{Prospettiva temporale}
Nel corso dell'ultimo decennio lo scenario del mercato dei \sw ha mostrato una sostanziale evoluzione. Se agli inizi degli anni 2000 si evidenziava una cospicua frammentazione dei fornitori, nel tempo hanno affermato il loro predominio i grandi produttori come \swname{IBM},\swname{Microsoft} e \swname{Oracle}.

Alcune aziende, come la \swname{FileNet}, che prima detenevano una rilevante porzione di mercato sono state assorbite da altre o, come nel caso \swname{Staffware} hanno perso il loro ruolo primario cedendo il passo a \inglese{competitor} di maggiori dimensioni. Sono invece emersi nuovi protagonisti come, ad esempio, \swname{Microsoft} e \swname{Oracle} che all'inizio del decennio detenevano una quota di mercato trascurabile.

\begin{figure}[H]
  \centering
  \includegraphics[width=.85\textwidth]{mercato_2000}
  \caption{Situazione del mercato fornitori nel 2000.}
  \label{fig:mercato2000}
\end{figure}

\begin{figure}[H]
  \centering
  \includegraphics[width=.85\textwidth]{mercato_2012}
  \caption{Situazione del mercato fornitori nel 2012.}
  \label{fig:mercato2012}
\end{figure}

Per quanto riguarda il volume degli investimenti da parte delle aziende nell'acquisizione di \sw BPM è possibile evidenziare un andamento fortemente positivo con la tendenza a mantenere un elevato ritmo di crescita per il futuro.

Il grafico riportato in \figurename~\ref{fig:trend} rappresenta l'andamento mondiale degli investimenti nel settore dell'innovazione tecnologica ed in particolare nei \sw BPM.

\begin{figure}[H]
  \centering
  \includegraphics[width=.85\textwidth]{markettrend}
  \caption{Situazione del mercato fornitori nel 2012.}
  \label{fig:trend}
\end{figure}

\subsection{Analisi geotopografica del mercato}% analisi fuffosa
Adottando un approccio territoriale alla segmentazione del mercato, la situazione che emerge rivela una diversificazione del comportamento della domanda fra l'area geografica delle Americhe, l'Europa considerata congiuntamente al Medio Oriente e all'Africa (EMEA) e l'Asia \cite{bea:bpm}. In particolare, considerando il volume degli investimenti, questo risulta decisamente più elevato nella regione americana, dove un investimento dell'entità di 289 milioni di dollari annui denota il carattere prioritario riconosciuto alle soluzioni BPM\@.

La regione europea si colloca al secondo posto con un valore di 137 milioni di dollari, seguita dall'Asia con una spesa pari a 70 milioni di dollari, come illustrato dalla \figurename~\ref{fig:map1}.

\begin{figure}[H]
  \centering
  \includegraphics[width=.85\textwidth]{worldmap}
  \caption{Entità degli investimenti in soluzioni BPM nelle diverse aree.}
  \label{fig:map1}
\end{figure}

La situazione delle quote di mercato relativa ai fornitori mostra un sostanziale allineamento con i dati esposti in precedenza, dal momento che la quota maggiore risulta essere detenuta ancora una volta dalle Americhe (58,3\%), mentre Europa e Africa si arrestano ad un livello del 27,6\% e nella regione asiatica il dato rilevato corrisponde al 14,1\%.

Tale dato può essere giustificato dal fatto che il mercato attira la specializzazione dei fornitori in questo settore dato il crescente investimento nelle stesse zone.

\begin{figure}[H]
  \centering
  \includegraphics[width=.85\textwidth]{worldmap2}
  \caption{Ripartizione delle quote di mercato nelle vendite di sistemi BPM.}
  \label{fig:map3}
\end{figure}

Un dato più interessante emerge tuttavia incrociando l'analisi sul piano geografico con la prospettiva temporale, considerando la distribuzione sul territorio del tasso di crescita annuale degli investimenti in soluzioni BPM\@. In questo caso, infatti, la regione che comprende Europa, Medio Oriente e Africa mostra un incremento più marcato pari al 45,4\% al di sopra delle Americhe che invece si collocano al 44,8\%.

Tale situazione è dovuta al fatto che il settore ha raggiunto un buon livello di maturità nelle Americhe mentre in Europa si tratta di un mercato ancora giovane. Per quanto riguarda, invece, l'estremo oriente, l'Asia e l'Oceania il tasso di crescita denota un incremento annuale nel quinquennio 2006-2011 del solo 37\%.

\begin{figure}[H]
  \centering
  \includegraphics[width=.85\textwidth]{worldmap3}
  \caption{Tasso di crescita annuo nel quinquennio 2006-2011 nel settore BPM.}
  \label{fig:map2}
\end{figure}

\section{Software selezionati}
	I \sw oggetto della presente analisi saranno considerati sulla base di una lista di caratteristiche significative delineate sulla base dei requisti esposti dalla \customer.
	
	Le funzionalità delle quali si terrà conto nell'analisi comparativa dei \sw sono:
	
\begin{enumerate}
\renewcommand{\labelenumi}{\arabic{enumi}}		
\renewcommand{\labelenumii}{\arabic{enumi}.\arabic{enumii}}
  \item modellazione
 	\begin{enumerate}
	  \item modellazione tramite editor grafico
	  \item rispetto di notazioni \inglese{standard}	
	\end{enumerate}
	\item monitoraggio dei processi
		\begin{enumerate}
		  \item sistema di \inglese{reporting}
		  \item indicatori grafici dello stato di avanzamento
		  \item ottimizzazione dei processi	
		\end{enumerate}		
	\item simulazione
		\begin{enumerate}
		  \item sistema di \inglese{reporting} simulazione
		  \item validazione modelli 	
		\end{enumerate}
	\item esecuzione automatizzata dei processi
	\item integrazione con \sw di parti terze
	\item interfacciamento con utenti
	\begin{enumerate}
	  \item \inglese{form} statici
	  \item \inglese{form} dinamici
	  \item \inglese{form} internazionalizzabili
	  \item gestione separata dei ruoli 			
	\end{enumerate}
\end{enumerate}

\subsection{Bonita BPM}\label{sec:bonita}
\newcommand{\progname}{\swname{Bonita\,BPM}\xspace}
\begin{picture}(0,0)
  \put(260, 10){\includegraphics[width=.3\textwidth]{bonitasoft_logo}}
\end{picture}

\subsubsection{Descrizione}
Il \sw è realizzato dalla \sw \inglese{house} \swname{BonitaSoft} e distribuito secondo un modello `open core', vale a dire le funzionalità di base sono rilasciate con una licenza \inglese{open source} e possono essere utilizzate e distribuite liberamente, mentre per avere accesso a funzionalità aggiuntive è necessario utilizzare estensioni proprietarie per cui è necessario acquisire una licenza d'uso dal produttore.

Il nucleo `open' dell'applicativo è costituito a sua volta in una serie di moduli grafici e da un motore di esecuzione dei processi.

I tre moduli grafici permettono, rispettivamente, di modellare i processi esistenti mediante un editor di flussi di lavoro (\swname{Bonita Studio}), di realizzare dei \inglese{form} personalizzati a supporto dello svolgimento delle attività codificate nell'editor dei processi (\swname{Bonita Form Builder}) e, infine, dal modulo \swname{Bonita User Experience} che permette di visualizzare per ciascun soggetto coinvolto nella realizzazione di un processo delle attività da svolgere.

Il motore di esecuzione permette invece, una volta creato e configurato il processo e ottenuto il \inglese{business archive} (BAR) corrispondente al diagramma eseguibile BPMN,\footnote{%
Il \inglese{Business Process Model and Notation} è un linguaggio \inglese{standard} di rappresentazione dei processi di \bsn che si inseriscono all'interno del modello di \bsn di un'azienda. La versione 2.0 dello \inglese{standard} introduce, accanto alla notazione per la rappresentazione visuale, il BPMN eseguibile che permette l'automazione dei processi.
}
di portarne a termine l'esecuzione provocando la visualizzazione dei \inglese{form} per l'input dei dati agli utenti finali coinvolti nelle operazioni.

\begin{figure}[H]
  \centering
  \includegraphics[width=.8\textwidth]{bonita_architecture}
  \caption{Architettura ad alto livello di \progname.}
  \label{fig:bonitaarchitecture}
\end{figure}

In \figurename~\ref{fig:bonitaarchitecture} è riportata un'illustrazione dell'architettura di sistema che si configura utilizzando \progname. In essa è possibile osservare i tre moduli che gestiscono l'interazione con gli utenti finali, il motore di esecuzione dei processi e il database sottostante nonché la relazione che intercorre fra quest'ultimo e i connettori per i servizi esterni necessari all'automatizzazione dei processi, allo scambio di informazioni e all'invio dei messaggi via \inglese{email}.

\subsubsection{Requisiti di sistema}
\begin{itemize}
  \item Java SE \inglese{Runtime Environment} (versione 6 o superiore) per l'esecuzione in locale;
  \item un \inglese{server} Java EE (ad esempio Tomcat versione 6.x o superiore) per il \inglese{deployment} dell'applicazione in remoto.
\end{itemize}

\subsubsection{Caratteristiche}
L'\inglese{editor} grafico offerto da \progname si presenta molto intuitivo e semplice da utilizzare. Un qualsiasi utente che abbia una minimale esperienza di \inglese{editor} grafici è in grado di cogliere ed utilizzare le funzionalità base. È chiaro, però, che per produrre diagrammi di buona qualità e per accedere a funzionalità più specifiche, è necessario consultare la documentazione. 

Inoltre, l'\inglese{editor} grafico offerto da \progname, rispetta lo \inglese{standard} BPMN 2.0 per la rappresentazione dei flussi di lavoro. L'uso dello \inglese{standard} è molto importante perché permette una comunicazione molto più completa e priva di ambiguità, anche con altri operatori del settore. Grazie all'interoperabilità tra strumenti di diversi fornitori garantita dal rispetto degli \inglese{standard}, è possibile diminuire gli eventuali costi di riconversione in una futura migrazione verso altre soluzioni.

Un esempio di un semplice diagramma nel formato \inglese{standard} BPMN è riportato in figura \ref{fig:diagramma}

\begin{figure}[H]
  \centering
  \includegraphics[width=.9\textwidth]{diagrammi}
  \caption{Diagramma prodotto con il \sw \progname.}
  \label{fig:diagramma}
\end{figure}

\progname offre la possibilità di installazione del \sw in lingua italiana come parte del pacchetto di base. Questa caratteristica risulta essere molto significativa in quanto molti \sw non dispongono di una localizzazione nativa e questo potrebbe rappresentare un fattore penalizzante in relazione ai requisiti del cliente. Inoltre, la localizzazione in lingua italiana risulta essere sufficientemente completa e di buona qualità.

Il sistema \progname comprende nel pacchetto di base, la funzionalità di simulazione dei processi. La simulazione è molto importante perché permette ai responsabili del processo decisionale, soprattutto se poco esperti, di avere una percezione sull'andamento ottimale dei processi. In questo modo si potrà monitorare e controllare la situazione possedendo un metro di confronto.
Si riporta in figura \ref{fig:simulazione} un estratto del \inglese{report} derivante dalla simulazione di un processo.

\begin{figure}[H]
  \centering
  \includegraphics[width=.9\textwidth]{simulazione}
  \caption{Estratto di un \inglese{report} di una simulazione con \progname.}
  \label{fig:simulazione}
\end{figure}

\progname è dotato della funzionalità di monitoraggio dei processi comprendente un sistema di \inglese{reporting} in grado di attivarsi automaticamente al termine del singoli processi e con possibilità di impostazione di tempistiche intermedie. Questo permette ai responsabili del processo decisionale di essere sempre informati, anche in modo automatico, sulla situazione vigente.

Il sistema di \inglese{reporting} è ben strutturato, restituisce un documento sia in formato \texttt{.pdf} che in \texttt{xls}. Tuttavia, il sistema fornisce il \inglese{report} in solo lingua inglese.

Il sistema supporta la funzionalità di interfacciamento con servizi LDAP\footnote{LDAP,  Lightweight Directory Access Protocol, è un protocollo \inglese{standard} per l'interrogazione e la modifica dei servizi di directory per la condivisione dei dati.\label{note:ldap}}. 

Questa funzionalità è richiesta dalla maggior parte delle aziende che fanno uso di \sw BPM\@. Infatti, tale funzione, permette loro di gestire la propria documentazione in modo semplice ed efficace centralizzando l'accesso ai documenti che costituiscono il flusso informativo aziendale.

Inoltre, \progname è in grado di interfacciarsi tramite l'utilizzo di appositi connettori anche alla gestione di contenuti collaborativi offerta da \swname{Microsoft SharePoint} nonché con il sistema di gestione della posta elettronica \swname{Microsoft Exchange} di cui dispone \customer.

La condivisione delle risorse non fa parte del pacchetto base di \progname. Tuttavia l'azienda può usufruirne tramite l'installazione del pacchetto \textsf{teamwork}, dietro un minimale corrispettivo. Anche questa funzionalità è molto richiesta dalle aziende. Si tratta di una necessità molto risentita in quanto per loro stessa natura i progetti coinvolgono una pluralità di soggetti che hanno l'esigenza di  operare sinergicamente da diverse postazioni.

La versione base del \sw dispone, inoltre, del modulo \swname{User Experience}. Tale modulo permette all'azienda che adotta \swname{Bointa BPM}, di disporre di due tipologie di interfacciamento:
\begin{itemize}
 \item utente semplice
 \item amministratore
\end{itemize}

In questo modo l'utente semplice, specialmente se inesperto, non disponendo di alcuni permessi, non potrà causare eventuali danni che minerebbero il buon funzionamento del \sw.

Inoltre, le due interfacce permettono di gestire in maniera adeguata il compito dei diversi ruoli. Infatti, mentre l'interfaccia utente permette di eseguire \inglese{task} e comunicarne, sempre tramite il \sw, lo stato intermedio o la terminazione, l'interfaccia amministrativa permette ai responsabili decisionali, invece, di monitorare la situazione, essendo loro informati sullo stato di ogni \inglese{task} e di effettuare delle modifiche alla pianificazione laddove lo ritengano necessario.
Si riporta nella figura \ref{fig:utente} un esempio dell'interfaccia utente e in figura \ref{fig:amministratore} un esempio dell'interfaccia amministrativa.
\begin{figure}[H]
  \centering
  \includegraphics[width=.9\textwidth]{utente}
  \caption{Interfaccia utente di \progname.}
  \label{fig:utente}
\end{figure}

\begin{figure}[H]
  \centering
  \includegraphics[width=.9\textwidth]{amministratore}
  \caption{Interfaccia da amministratore di \progname.}
  \label{fig:amministratore}
\end{figure}

Per quanto riguarda la documentazione fornita dal \inglese{team} di sviluppo, fornisce spiegazioni molto esaustive e comprende anche degli ottimi \inglese{tutorial} di supporto. 
Tuttavia, la documentazione è disponibile solo nelle versione inglese. Questo potrebbe causare dei problemi all'azienda acquirente qualora non disponga di personale con una sufficiente preparazione linguistica.

\progname, purtroppo, risulta non avere una facile installazione e inoltre la predisposizione di tutti i moduli e, delle eventuali versioni in abbonamento, risulta essere complessa. Si prevede quindi la necessità, in caso di scelta di questo \sw, di disporre di una figura atta all'installazione del sistema in azienda. 

Occorre considerare che, molte funzionalità aggiuntive potenzialmente molto interessanti, sono a pagamento, in particolare:
\begin{itemize}
  \item l' accesso a funzionalità collaborative (\inglese{repository});
  \item controllo sullo stato di avanzamento delle sotto-attività e sull'utilizzo delle risorse;
  \item interfacciamento con applicativi gestionali \swname{SAP} tramite gli appositi connettori.
\end{itemize}

Infine, anticipato in precedenza, \progname comprende la funzionalità di esecuzione automatica dei processi. Infatti, tramite il \inglese{deployment} dei processi è possibile automatizzarne l'esecuzione delle componenti che non devono essere gestite da risorse umane.

Per concludere, in Tabella~\ref{tab:bonitavers} è riportato un prospetto riassuntivo delle funzionalità che sono disponibili in \progname nella versione \inglese{open source} e sottoscrivendo i diversi abbonamenti.

\begin{small}
\begin{longtable}{>{\sffamily}p{.35\textwidth}*{4}{>{\sffamily}c}}
\toprule
\bfseries{}Funzionalità & \bfseries{}Community & \bfseries{}Teamwork & \bfseries{}Efficiency & \bfseries{}Performance\\
\midrule
modellazione processi in BPMN2                     & \tick  & \tick  & \tick  & \tick \\
simulazione dei processi                           & \tick  & \tick  & \tick  & \tick \\
connettori a sistemi esterni                       & \tick  & \tick  & \tick  & \tick \\
generazione \inglese{web forms} semplici           & \tick  & \tick  & \tick  & \tick \\
generazione \inglese{web forms} dinamici           & \cross & \tick  & \tick  & \tick \\
\inglese{debugger} dei processi                    & \tick  & \tick  & \tick  & \tick \\
profili utente personalizzati                      & \cross & \tick  & \tick  & \tick \\
generazione documentazione                         & \cross & \tick  & \tick  & \tick \\
gestione documenti con \inglese{versioning}        & \cross & \tick  & \tick  & \tick \\
\inglese{repository} BPM collaborativi             & \cross & \tick  & \tick  & \tick \\
ottimizzazione dei processi                        & \cross & \tick  & \tick  & \tick \\
profili utente personalizzabili                    & \cross & \cross & \tick  & \tick \\
motore di \inglese{deployment} dei processi        & \tick  & \tick  & \tick  & \tick \\
\inglese{deployment} remoto                        & \cross & \tick  & \tick  & \tick \\
gestione utenti                                    & \tick  & \tick  & \tick  & \tick \\
gestione privilegi                                 & \cross & \tick  & \tick  & \tick \\
internazionalizzazione dei \inglese{form}          & \cross & \tick  & \tick  & \tick \\
sincronizzazione LDAP                              & \cross & \tick  & \tick  & \tick \\
monitoraggio risorse                               & \cross & \cross & \cross & \tick \\
monitoraggio processi                              & \cross & \cross & \cross & \tick \\
gestione errori                                    & \cross & \cross & \cross & \tick \\
\bottomrule
\caption{Riassunto delle differenze fra le versioni di \progname.}
\label{tab:bonitavers}
\end{longtable}
\end{small}

\subsection{ProcessMaker}
\renewcommand{\progname}{\swname{ProcessMaker}\xspace}
\begin{picture}(0,0)
  \put(270, 10){\includegraphics[width=.3\textwidth]{processmaker_logo}}
\end{picture}

\subsubsection{Descrizione}
\progname è la soluzione \inglese{open source} sviluppata dalla società statunitense \swname{Colosa} per la modellazione e il monitoraggio dei processi aziendali, che si affianca ad altri prodotti proprietari con funzionalità più estese come \swname{ProcessMapper}.

Il \sw è completo delle funzionalità necessarie a una PMI come \customer e può contare sul supporto di una grande comunità di utenti che fornisce una vasta documentazione, moduli di localizzazione e \inglese{plug-in}. Inoltre, per il suo funzionamento si basa su piattaforme \inglese{standard} e diffuse come il \inglese{web server} \swname{Apache}, database \swname{MySQL} e l'interprete \swname{PHP}.

\subsubsection{Requisiti di sistema}	
\begin{itemize}
	\item installazione di un \inglese{web server} \swname{Apache} versione 2.2.x con abilitati i moduli necessari (\texttt{deflate}, \texttt{expires}, \texttt{rewrite} e \texttt{vhost\_alias}) e la pagina di \inglese{default} disabilitata;
	\item presenza dell'interprete \swname{PHP} versione 5.1.6  (o superiore) con installati i moduli obbligatori \texttt{mysql}, \texttt{xml} e \texttt{curl};
	\item \inglese{Relational Database Management System} \swname{MySQL} versione 5 o superiore installato nello stesso \inglese{server} in cui risiede \progname.
\end{itemize}

\subsubsection{Caratteristiche}
L'architettura basata su un \inglese{server} cui si accede tramite un'interfaccia \inglese{web-based} permette di svincolare in maniera pressoché totale la piattaforma \sw responsabile della memorizzazione dei dati e dell'esecuzione della logica applicativa dai sistemi \inglese{client} che la utilizzeranno, facendo uso di tecnologie estremamente diffuse come \swname{Apache}, \swname{MySQL} e \swname{PHP} che costituiscono la piattaforma AMP.

Nonostante tale soluzione sia declinabile anche in versione \swname{Microsoft Windows} (WAMP), tuttavia, rimane il problema di gestire la coesistenza fra \swname{Internet Information Services (IIS)} presente nell'ambiente \swname{Microsoft Small Business server} e il \inglese{web server} \swname{Apache}, che dovrà essere affidata a personale dedicato e in possesso delle necessarie qualifiche.

Inoltre, la procedura di installazione di tutti i servizi citati nei requisiti di sistema può risultare complessa in quanto non tutti i moduli richiesti sono disponibili nelle installazioni tipiche e, nonostante il \inglese{wizard} di configurazione offerto da \progname semplifichi la procedura, questa può risultare comunque ostica per i non addetti ai lavori.

Per quanto concerne l'interfacciamento con applicativi di terze parti, \progname nativamente non dispone di funzionalità degne di nota ma è possibile ricorrere a una serie di \inglese{plug-in} esterni. In tal modo, ad esempio, è possibile integrare la soluzione con piattaforme per la condivisione di risorse basate su LDAP (cfr.~nota~\ref{note:ldap}) o per la gestione del materiale documentale con \swname{KnowledgeTree}.

Un'ulteriore possibilità di estensione ottenibile tramite \inglese{plug-in} è l'integrazione con \swname{Microsoft Outlook} per gestire automaticamente l'invio delle notifiche ai collaboratori coinvolti nei processi aziendali. \progname consente infatti la gestione automatizzata dei processi che sono stati modellati attraverso l'editor grafico e l'invio delle notifiche	via \inglese{email} ai collaboratori.

Per quanto riguarda l'editor di diagrammi, riportato in \figurename{fig:pmeditor}, sono da segnalare alcune carenze piuttosto rilevanti. In primo luogo, il modello di rappresentazione visuale dei processi utilizza una sintassi simile ai diagrammi di flusso o ai diagrammi di attività UML ma non rispetta lo \inglese{standard} BPMN e questo pone seri limiti all'interoperabilità con applicativi di terze parti.

\begin{figure}[H]
  \centering
  \includegraphics[width=.9\textwidth]{pm_editor}
  \caption{L'editor visuale dei processi interno a \progname.}
  \label{fig:pmeditor}
\end{figure}

In secondo luogo, non sono presenti alcune funzionalità visuali come l'inserimento automatico delle guardie per le condizioni di \inglese{branch} (nodi decisionali) e per i nomi dei nodi che rappresentano attività, per cui è necessario configurare le proprietà manualmente in un secondo momento a valle dell'inserimento.

L'interfaccia \inglese{web} costringe inoltre a ricorrere estensivamente alla \inglese{gesture} `\inglese{drag and drop}' per la creazione di diagrammi, caratteristica che non può essere considerata positiva alla luce dei più recenti studi di usabilità in quanto presuppone una maggiore difficoltà di utilizzo e di discosta dalle buone prassi per la realizzazione di interfacce utente \cite{nielsen:mistakes}.

Inoltre, l'assenza di una funzionalità di \inglese{undo} rappresenta una grave carenza funzionale in quanto, a causa dei limiti dell'interfaccia utente descritti in precedenza, risulta molto facile per gli analisti di \bsn perdere il \inglese{focus} dell'attenzione e, conseguentemente, commettere errori. Infine, la creazione dei diagrammi nell'editor non è intuitiva.

Si segnala, d'altro canto, che la localizzazione in lingua italiana non è offerta nativamente ma solo come modulo integrativo e che il file PO (\inglese{Portable Object}) messo a disposizione dalla comunità di utilizzatori di \progname contiene tutte le stringhe per la traduzione dell'interfaccia, con il conseguente risultato che alcune voci di menu non contengono alcuna etichetta testuale, in quanto non viene visualizzata nemmeno la stringa originale.

Un aspetto fortemente positivo dell'interfaccia grafica di \progname è costituito dalla visualizzazione dello stato di avanzamento dei processi tramite una metafora visiva che ricorda esplicitamente gli indicatori presenti sul quadro strumenti di un macchinario o di un veicolo (vedi \figurename~\ref{fig:merdometro}).

\begin{figure}[H]
  \centering
  \includegraphics[width=.9\textwidth]{merdometro}
  \caption{Indicatore della misura di avanzamento del lavoro in \progname.}
  \label{fig:merdometro}
\end{figure}

Tale accorgimento ha il vantaggio di fornire una rappresentazione visuale immediata ed efficace della misura dell'indicatore di \inglese{performance} $i_{t}$ definito come il rapporto:
\[
i_{t} = \frac{N^{C}_{t}}{N^{T}}
\]
dove $N^{C}_{t}$ rappresenta il numero di \inglese{task} completati all'istante $t$ e $N^{T}$ il numero totale di \inglese{task} in cui è stato complessivamente decomposto il processo. Il \sw rende quindi possibile ai responsabili decisionali valutare a colpo d'occhio la situazione e stabilire se è richiesta l'adozione di misure correttive in tempo reale (`latenza zero', cfr.~sez.~\ref{sec:currentmarket}).

Per permettere la gestione e la ripartizione del lavoro tra i soggetti coinvolti in ambito aziendale, \progname premette alle figure di \bsn \inglese{analyst} in qualità di amministratori la creazione di profili degli utenti e di gruppi di appartenenza degli utenti stessi a cui possono essere assegnati i \inglese{task} di cui si compongono i processi, con una \inglese{dashboard} di amministrazione esemplificata in \figurename~\ref{fig:pmuserlist}.

\begin{figure}[H]
  \centering
  \includegraphics[width=.9\textwidth]{pm_userlist}
  \caption{Interfaccia di amministrazione dei profili utente e gruppi.}
  \label{fig:pmuserlist}
\end{figure}

Il monitoraggio dei \inglese{case}, vale a dire delle istanze dei processi aziendali in esecuzione, prevede di \inglese{default} la visualizzazione della durata e il numero di \inglese{case} per processo. Tali \inglese{report} di base possono però essere ampliati mediante l'installazione di appositi \inglese{plug-in}.

All'accesso al sistema, gli utenti cui sono state assegnate le attività possono visualizzare l'elenco dei rispettivi compiti all'interno del \inglese{case} corrente tramite una schermata simile a quella riportata in \figurename~\ref{fig:pmtasklist}. 

\begin{figure}[H]
  \centering
  \includegraphics[width=.9\textwidth]{pm_dashboard}
  \caption{Visualizzazione della lista dei \inglese{task} di competenza di un utente.}
  \label{fig:pmtasklist}
\end{figure}

L'interazione degli utenti finali, cioè del personale coinvolto nello svolgimento delle attività coordinate di cui si compongono i singoli processi, utilizzando \progname passa anche e soprattutto un'interfaccia \inglese{web} all'interno della quale è prevista la compilazione di \inglese{form} dinamici che per il \inglese{team} di sviluppo assumono il nome \swname{Dynaforms}.

La caratteristica fortemente positiva da questo punto di vista è rappresentata dalla facilità d'uso ma anche di realizzazione di questi \inglese{form} tramite modelli predefiniti messi a disposizione dal sistema.

La visualizzazione dell'interfaccia può anche essere personalizzata a livello di gruppo di utenti -- una volta applicata a un gruppo l'impostazione avrà effetto su tutti i suoi membri -- nelle versioni \textsf{Normal}, \textsf{Simplified} (che dispone di un sottoinsieme delle funzionalità) o \textsf{Switchable} (che permette di alternare fra le due modalità precedentemente descritte). Non si tratta, tuttavia, di una vera e propria personalizzazione dell'interfaccia sulla base del ruolo ricoperto dall'utente.
	
\subsection{Bizagi}
\renewcommand{\progname}{\swname{Bizagi}\xspace}
\begin{picture}(0,0)
  \put(290, 10){\includegraphics[width=0.2\textwidth]{bizagi_logo}}
\end{picture}

\subsubsection{Descrizione}
La soluzione \inglese{open source} sviluppata dalla società inglese \swname{Bizagi} si compone di due moduli:
\begin{itemize}
  \item \swname{Bizagi Process Modeler}, editor per la modellazione dei processi 
  \item \swname{Bizagi Suite}, \sw per la gestione ed il monitoraggio dei processi aziendali
\end{itemize}

La soluzione offerta da \progname è completa ed è sicuramente alla portata di una PMI come \customer. Si tratta infatti di un \sw particolarmente performante in grado di fornire informazioni molto precise sull'andamento dei processi.

\subsubsection{Requisiti di sistema}
\begin{itemize}
  \item sistemi operativi \swname{Microsoft Windows}
  \item \swname{SQL Server Express 2008 SP3} e successivi
\end{itemize}

\subsubsection{Caratteristiche}
L'\inglese{editor} grafico è sufficientemente intuitivo e rispetta la notazione BPMN 2.0. Inoltre, l'interfaccia grafica ha un \inglese{look and feel} consistente con la applicazioni del sistema \swname{Windows}. Questo permette alle aziende che decidono di acquisire \progname di ritrovarsi in un contesto già conosciuto e quindi la curva di apprendimento diventa meno ripida.
Si riporta in figura \ref{fig:diagramma_biz} un esempio di diagramma prodotto con \swname{Bizagi Process Modeler}.

\begin{figure}[H]
  \centering
  \includegraphics[width=.9\textwidth]{diagramma_biz}
  \caption{Esempio di diagramma creato con  \swname{Bizagi Process Modeler}}
  \label{fig:diagramma_biz}
\end{figure}

L'\inglese{editor} comprende anche la funzionalità di verifica dei diagrammi, che permette di fare il  \inglese{debugging} sui grafici prodotti.
In figura \ref{fig:verifica} è riportato il messaggio restituito dal programma in seguito alla verifica effettuata.

\begin{figure}[H]
  \centering
  \includegraphics[width=.9\textwidth]{verifica}
  \caption{Messaggio restituito dalla funzionalità di verifica di \swname{Bizagi Process Modeler}}
  \label{fig:verifica}
\end{figure}
Per quanto riguarda invece, la gestione della lingua, l'italiano è offerto come funzionalità nativa del \sw \progname. Tuttavia, la localizzazione in lingua italiana non è completa e diverse parti risultano essere tradotte male o non tradotte affatto.

\progname permette la simulazione dei processi. In seguito ai test eseguiti, però, risulta essere più deludente rispetto a quella offerta da altri \sw presi in considerazione nel corso della presente analisi.
I \inglese{report} restituiti dalla simulazione risultano essere poco comprensibili, incompleti e non contengono alcun grafico, il quale, permetterebbe ai responsabili decisionali di avere una realizzazione molto più veloce della situazione.
 
La documentazione fornita dal \inglese{team} di sviluppo è completa e comprende anche dei video \inglese{tutorial} di supporto. Purtroppo, sia la documentazione che i \inglese{tutorial} sono forniti solo in lingua inglese. Inoltre, \progname mette a disposizione dei \inglese{process template}, che permettono agli utenti non esperti di avere la concezione delle potenzialità del programma.

L'interfaccia offerta da \swname{Bizagi Suite} è particolarmente attraente e performante, ma purtroppo poco intuitiva. L'utente, soprattutto se inesperto, fa molta fatica a capire come utilizzare l'applicativo. Anche se la gestione dei processi non è per nulla \inglese{user friendly}, il \sw offre molti strumenti di \inglese{management} e permette di ricavare molte informazioni sull'andamento dei processi.
Inoltre, non esistono diverse tipologie di interfacce in relazione ai ruoli ricoperti, funzionalità offerta, invece, da molti altri \sw BPM.

Si riporta in figura \ref{fig:suite} un immagine dell'interfaccia di \swname{Bizagi Suite}

\begin{figure}[H]
  \centering
  \includegraphics[width=.9\textwidth]{suite}
  \caption{Interfaccia di \swname{Bizagi Suite}}
  \label{fig:suite}
\end{figure}

\progname offre un ottimo sistema di \inglese{reporting}. I dati forniti sono molto dettagliati e la documentazione è disponibile in diversi formati.

\progname supporta l'integrazione con altre applicazioni, in particolare, tra le più importanti troviamo l'integrazione con gli \inglese{Enterprise Content Management} (ECM) come \swname{Sharepoint} e con gestionali \inglese{Customer Relationship Management} (CRM).

Come indicato tra i requisiti di sistema, \progname può essere installato solo su piattaforme \swname{Windows}. Questo, in generale, costituisce un limite, tuttavia, in questo caso, visto che \customer dispone di un'infrastruttura \swname{Windows}-\inglese{based}, non vi è alcun problema.

Infine, \swname{Bizagi Process Modeler} comprende una \inglese{feature} che permette a più utenti di collaborare contemporaneamente alla modellazione di un processo.  Si tratta di una funzionalità molto interessante, soprattutto in \inglese{team} numerosi, per parallelizzare il lavoro e quindi aumentare la produttività.

\subsection{Camunda BPM Platform}
\renewcommand{\progname}{\swname{Camunda}\xspace}
\begin{picture}(0,0)
  \put(260, 10){\includegraphics[width=.3\textwidth]{camunda_logo}}
\end{picture}

\subsubsection{Descrizione}
\progname è una soluzione realizzata da un \inglese{team} di sviluppatori con sede a Berlino e distribuita sotto licenza \inglese{Apache v2.0} e quindi totalmente \inglese{open source}. Il \sw si basa sulla piattaforma Java e sul \inglese{framework} per la realizzazione di soluzioni BPM \swname{Activiti}. 

Nella sua distribuzione di base comprende un motore di esecuzione dei processi (`\inglese{process engine}') e tre \inglese{front-end} grafici (`\inglese{process applications}') indipendenti: \swname{Tasklist} utilizzato per la gestione dei flussi di lavoro basati sull'interazione con gli utenti umani, \swname{Cockpit} specializzato nella gestione e il controllo delle operazioni aziendali e \swname{Cycle} che permette di importare e rendere eseguibili i modelli di processi BPMN 2.0.

\subsubsection{Requisiti di sistema}
Al fine di eseguire il sistema è richiesta un'installazione di \swname{Apache Tomcat} quindi sono in particolare necessari:
\begin{itemize}
  \item il \inglese{servlet container} \swname{Catalina};
  \item il connettore HTTP \swname{Coyote};
  \item il motore Java Server Pages \swname{Jasper};
\end{itemize}
tuttavia, nella sua distribuzione \inglese{standard}, il \sw comprende anche gli eseguibili binari di un sistema \swname{Tomcat} con \progname già installato e configurato, più gli \inglese{script} di lancio automatico per i sistemi \swname{Windows}.

\subsubsection{Caratteristiche}
\progname non comprende alcun editor visuale per la modellazione dei processi aziendali al suo interno: tale funzionalità è demandata a \sw di terze parti specializzate a tale scopo (ne esistono anche di \inglese{open source} come, ad esempio, \swname{Yaoqiang}) ed essendo il BPMN un formato \inglese{standard} questo non rappresenta un problema in sede di importazione dei dati all'interno del sistema tramite \swname{Cycle}.

Quest'ultimo permette di creare un collegamento bidirezionale fra la rappresentazione in BPMN del processo di \bsn e il suo \inglese{deployment} nel sistema BPM sia attraverso il \inglese{file system} del server in cui è installato \progname che tramite un sistema di controllo versione (SVN) in modo da permettere ai \bsn \inglese{analyst} di intervenire anche da remoto utilizzando strumenti di modellazione in locale. In alternativa, è possibile utilizzare anche editor di modelli BPMN \inglese{cloud-based} (SaaS) come \swname{Signavio}, per cui \progname mette a disposizione un apposito connettore.

\swname{Cycle} permette inoltre di avere accesso al sistema di gestione delle identità che permette la creazione dei profili degli utenti (semplici o amministratori) del sistema e la loro organizzazione in gruppi che sia in grado di rispecchiare l'organigramma aziendale. La schermata di amministrazione degli utenti di \progname è riportata, a titolo esemplificativo, in \figurename~\ref{fig:comundauserlist}.

\begin{figure}[H]
  \centering
  \includegraphics[width=.9\textwidth]{camunda_userlist}
  \caption{Configurazione di utenti e gruppi in \swname{Camunda Cycle}.}
  \label{fig:comundauserlist}
\end{figure}

\progname permette inoltre la creazione di \inglese{form} personalizzati per la gestione automatizzata dell'interazione degli utenti finali con il sistema BPM\@. A questo proposito sono distinte due categorie di moduli: i cosiddetti \inglese{start form} visualizzati da parte dell'utente prima che l'istanza di esecuzione del processo sia stata effettivamente avviata e i \inglese{task form} che invece sono mostrati nelle fasi intermedie e sono finalizzati al completamento dei \inglese{task} di cui sono composte le attività di processo.

L'esecuzione dei processi che prevedono l'intervento di attori umani è realizzata attraverso l'applicazione \swname{Tasklist} attraverso la quale è possibile accedere alla lista di compiti che sono assegnato a un determinato utente e possono essere dis-assegnati o delegati ad altri colleghi. Inoltre, all'interno dell'applicazione di interfacciamento sono visualizzati i \inglese{form} sia di tipo \inglese{start} che di tipo \inglese{task}, come esemplificato in \figurename~\ref{fig:camundatasklist}

\begin{figure}[H]
  \centering
  \includegraphics[width=.9\textwidth]{camunda_tasklist}
  \caption{Visualizzazione di un \inglese{task form} all'interno di \swname{Camunda Tasklist}.}
  \label{fig:camundatasklist}
\end{figure}

Il monitoraggio dello svolgimento dei processi integra una funzionalità di \inglese{reporting} di alcune categorie particolari di eventi che possono verificarsi nello svolgimento dei processi (\inglese{incidents}), come ad esempio il fallimento di un'attività interna.

Lo strumento principale per il monitoraggio dei processi all'interno della \inglese{suite} \progname è la `cabina di pilotaggio' \swname{Cockpit}. Grazie a quest'ultimo, è possibile verificare il numero di istanze attive per ciascun processo in un determinato momento e, per ognuna delle istanze, il grado di completamento (qual è l'attività attualmente in esecuzione) e, per le attività già svolte, quali sono stati i dati inseriti nei \inglese{form}, come illustrato dalla schermata riportata in \figurename~\ref{fig:camundacockpit}.

\begin{figure}[H]
  \centering
  \includegraphics[width=.9\textwidth]{camunda_cockpit}
  \caption{Monitoraggio dell'avanzamento di un processo in \swname{Camunda Cockpit}.}
  \label{fig:camundacockpit}
\end{figure}

\section{Analisi comparativa}
\begin{small}
\begin{longtable}{>{\sffamily}p{.4\textwidth}*{4}{>{\sffamily}c}}
\toprule
\bfseries{}Funzionalità & \bfseries{}Bonita\,BPM & \bfseries{}ProcessMaker & \bfseries{}Bizagi & \bfseries{}Camunda\\
\midrule
modellazione                                 & \tick  & \tick  & \tick      & \cross \\
modellazione tramite editor grafico          & \tick  & \tick  & \tick     & \cross \\
rispetto notazione BPMN                      & \tick  & \cross &  \tick     & \cross \\
importazione/esportazione file XPDL          & \tick  & \tick &  \tick     & \cross \\
monitoraggio dei processi                    & \tick  & \tick  &  \tick    & \tick \\
sistema di \inglese{reporting}               & \tick  & \tick  &  \tick    & \tick \\
indicatori grafici stato di avanzamento      & \cross & \tick  &   \cross   & \cross \\
ottimizzazione dei processi                  & \tick$^{*}$ & \cross & \cross      & \cross \\
simulazione                                  & \tick  & \cross &  \tick       & \cross \\
sistema di \inglese{reporting} simulazione   & \tick  & \cross &  \tick      & \cross \\
validazione modelli                          & \tick  & \cross &  \tick     & \cross \\
esecuzione automatizzata processi            & \tick  & \tick  &   \tick & \tick \\
integrazione con \sw di parti terze          & \tick  & \tick  &   \tick     & \tick \\
interfacciamento con utenti                  & \tick  & \tick  &    \tick    & \tick \\
\inglese{form} statici                       & \tick  & \tick  &    \tick    & \cross \\
\inglese{form} dinamici                      & \tick$^{*}$ & \tick  & \cross        & \cross \\
\inglese{form} internazionalizzabili         & \tick$^{*}$ & \cross &   \cross     & \cross \\
gestione separata dei ruoli                  & \tick$^{*}$  & \cross & \cross       & \tick \\
\bottomrule
\begin{minipage}[b]{.8\textwidth}
\begin{tabular}{lp{\textwidth}}
(*) & i campi contrassegnati con questo simbolo si riferiscono a funzionalità che non sono disponibili nelle versioni di base dei programmi considerati ma solo nelle versioni a pagamento o previa sottoscrizione di un abbonamento.\\
\end{tabular}
\end{minipage}
\end{longtable}
\end{small}

\chapter{Proposta}
\section{Introduzione}
In tale sezione sarà indicata la soluzione \sw BPM selezionata da \team per \customer e saranno spiegate le motivazioni alla base di tale scelta. Quest'ultima è scaturita da un'attenta analisi delle migliori soluzioni \sw disponibili in commercio e da una comparazione tra le diverse alternative presenti nel mercato.

\team ha privilegiato soluzioni \inglese{open source} con lo scopo di minimizzare i costi sia dell'attività di consulenza stessa che dell'effettivo \inglese{deployment} dell'applicazione nell'ambiente di produzione di \customer. Inoltre, pur essendo state vagliate soluzioni \inglese{made in Italy}, non è stato possibile testare i \sw dal momento che la maggior parte delle aziende adottano una politica di licenza proprietaria.

\section{Soluzione proposta}
\renewcommand{\progname}{\swname{Bonita\,BPM}\xspace}
\begin{picture}(0,0)
  \put(260, 15){\includegraphics[width=.3\textwidth]{bonitasoft_logo}}
\end{picture}
In seguito all'analisi effettuata \team propone l'acquisizione di \hyperref[sec:bonita]{\progname}. Tale scelta è stata determinata da diversi fattori. In primo luogo il \sw è in grado di soddisfare tutti i requisiti posti da \customer tramite l'utilizzo della sola versione di base che è disponibile tramite una licenza \inglese{open source} e in forma totalmente gratuita.

L'acquisizione di un \sw con licenza \inglese{open source} permette di introdurre le metodologie di BPM all'interno del modello di \bsn di \customer minimizzando i costi. Si evidenza inoltre che il \sw scelto può essere esteso con funzionalità di più alto livello tramite la sottoscrizione di abbonamenti a pagamento.

Qualora invece l'azienda volesse passare in seguito a un \sw con maggiori prestazioni con una licenza non totalmente gratuita, sarà comunque possibile sfruttare le competenze acquisite nell'utilizzo di \progname in quanto, in generale, i \sw BPM presentano una configurazione simile. 

\progname dispone infatti delle funzionalità di modellazione che permette di rappresentare i processi aziendali dando la possibilità  di superare le problematiche legate alla non linearità rilevate da \team. Tale rappresentazione è inoltre conforme allo \inglese{standard} BPMN e questo, garantendo l'interoperabilità, ha il vantaggio di minimizzare i costi di riconversione ad un'eventuale altra soluzione.

Inoltre il \sw offre un'ottima gestione dei processi e permette il controllo dello stato di avanzamento degli stessi e integra al suo interno utili funzionalità di simulazione dei processi aziendali e di validazione del modello di \bsn adottato. \progname dispone della funzionalità di esecuzione automatizzata dei processi che permette all'azienda di aumentare la propria efficienza riducendo i tempi di produzione. Il \sw è dotato inoltre di un ottimo sistema di \inglese{reporting} in grado di fornire dettagliate informazioni sullo stato dei processi.

Qualora il \inglese{report} evidenziasse delle problematiche, \progname è in grado di fornire in automatico delle indicazioni volte all'ottimizzazione dei processi che si trovano in stati critici assistendo gli organi decisionali interni ad \customer.

La gestione delle attività avviene tramite un portale che presenta interfacce differenti in relazione ai ruoli ricoperti dagli utilizzatori che possono essere utenti semplici oppure amministratori incaricati del monitoraggio. Tale funzionalità non è offerta da tutti i \sw presenti nel mercato che spesso non operano alcuna distinzione fra le categorie di utilizzatori.

La gestione invece dell'interazione con gli esecutori dei processi avviene mediante \inglese{form} che possono essere creati all'interno di \progname tramite il modulo \swname{Bonita Form Builder} sia in forma statica che in forma dinamica.

Per ulteriori approfondimenti relativi ai pregi e ai difetti rilevati in \progname si rimanda alla sezione \ref{sec:bonita} contenuta nello studio di mercato.

\bigskip
\noindent{}{\bfseries\sffamily{}Soluzione ibrida/soluzione ad hoc}\\
In seguito ad un'attenta valutazione \team ha deciso di escludere la possibilità di creare soluzioni ibride o \inglese{ad hoc}. Tale scelta è stata determinata dalla presenza di numerosi \sw già di per sé completi che non giustificherebbe l'ingente esborso di denaro necessario allo sviluppo di una nuova soluzione.

Lo sviluppo di una soluzione ibrida non è stato ritenuto necessario perché, anche qualora i \sw non fossero completi di \inglese{editor}, questi ultimi sono già predisposti per l'interazione con applicazioni specializzate a tale scopo.

Invece, per quanto riguarda lo sviluppo di una soluzione dedicata, il costo necessario per la realizzazione del prodotto non sarebbe stato sostenibile da \customer né necessario viste le esigue dimensioni dell'azienda.

\section{Costo finale}
Si riporta un riepilogo delle spese per la consulenza effettuata da \team per \customer:

\begin{center}
\small
\begin{tabular}{p{.5\textwidth}r}
\toprule
\bfseries\sffamily{}Attività & \bfseries\sffamily{}Costo\\
\midrule
\scshape{} pianificazione    & \EUR{1.500,00}\\
\scshape{} studio di mercato & \EUR{2.450,00}\\
\scshape{} proposta          & \EUR{905,00}\\
\midrule
Totale                       & \textcolor{red}{\EUR{4.855,00}}\\
\bottomrule
\end{tabular}
\end{center}

Il costo della prestazione di consulenza ammonta pertanto a \textbf{\EUR{4.855,00}}. Pur essendo emerse delle problematiche nella realizzazione del progetto \team si è impegnata a mantenere inalterato il preventivo stabilito apportando come unica variazione la decurtazione del costo di redazione del manuale pari a \EUR{60,00} che non è stato realizzato per i problemi riscontrato. A tale proposito, si rimanda alla documentazione \inglese{online} e si consiglia, eventualmente, data la particolare completezza del manuale, di rivolgersi a un'agenzia di traduzioni.

Il dettaglio dei costi è esposto nella sezione \ref{sec:aspettoeconomico} del presente documento.

\begin{thebibliography}{90}
\addcontentsline{toc}{chapter}{\bibname}
  \bibitem[Mazzolari, 2007]{mazzolari:bpmita} Mazzolari, F. e Testolin, G., \emph{Document Management}, SistemiNEWS, 2007,\newline disponibile all'indirizzo: \url{http://www.arxivar.it/download/doc_download/99-e-document-management} (consultato il 11/07/2013)
  \bibitem[Bea, 2008]{bea:bpm} BEA Systems, \inglese{The State of the BPM Market}, 2008, \newline disponibile all'indirizzo \url{http://www.inst-informatica.pt/servicos/informacao-e-documentacao/dossiers-tematicos/dossier-tematico-no-6-bpm-business-process/the-state-of-the-bpm-market-businessand-it-solving} (consultato il 12/07/2013)
  \bibitem[Nielsen, 2008]{nielsen:mistakes} Nielsen, J. \inglese{Top 10 Application-Design Mistakes}, \newline disponibile all'indirizzo: \url{http://www.nngroup.com/articles/top-10-application-design-mistakes} (data di consultazione:  15/07/2013)
\end{thebibliography}

\end{document}
