\documentclass[a4paper,11pt]{letter}
\usepackage[T1]{fontenc}
\usepackage[utf8]{inputenc}
\usepackage{lmodern}
\usepackage[italian]{babel}
\usepackage[osf]{mathpazo}

%*******************************************************************************
% Dati per la lettera
%*******************************************************************************
\address{}
\signature{Diego Beraldin\\Elena Zecchinato}
\date{Padova, 19 luglio 2013}

%*******************************************************************************
% Altri pacchetti utili
%*******************************************************************************
\usepackage{xspace} % per spazi condizionali

%*******************************************************************************
% Un po' di macro
%*******************************************************************************
\newcommand{\giraldo}{Ghiraldo\xspace}
\newcommand{\inglese}[1]{\foreignlanguage{english}{\textit{#1}}}
\newcommand{\team}{\textsf{EtaBeta\,Software}\xspace}
\newcommand{\customer}{\textsf{Alpha\,\&\,Partners}\xspace}

%*******************************************************************************
% Inizio del documento
%*******************************************************************************
\begin{document}

\begin{letter}{
  \textbf{Alla cortese attenzione di:} prof. Filippo \giraldo\\
  \textbf{Oggetto:} presentazione progetto}

\opening{Egregio prof. \giraldo,}

con la presente Le consegniamo l'elaborato svolto per l'esame di Sviluppo e Gestione Progetti. Il lavoro è composto da due documenti:
\begin{itemize}
\renewcommand{\labelitemi}{--}
  \item `Progetto SGP' (\texttt{progetto\_sgp.pdf});
  \item `Business Plan' (\texttt{business\_plan.pdf}).
\end{itemize}

Ai fini della redazione del primo documento, il \inglese{team} ha immaginato di essere \team, una società esterna alla quale \customer si è rivolta per una consulenza in merito all'acquisto di un \inglese{software} BPM.

Per realizzare il secondo documento, invece, il \inglese{team} ha immaginato di essere l'azienda \customer stessa e di redigere un \inglese{business plan} per il progetto di sviluppo interno ad \customer. Il \inglese{team} ha inoltre supposto che \team collaborasse con \customer per la stesura della parte relativa all'acquisizione del \inglese{software} BPM di detto documento.

Per quanto riguarda il consiglio di privilegiare soluzioni \inglese{made in Italy}, il \inglese{team} ha vagliato diversi prodotti tra cui \textsf{Arxivar}, \textsf{NESP} e \textsf{Sherpa}. Purtroppo non è stato possibile testare questi \inglese{sofware} non avendo avuto alcuna risposta da parte delle aziende che erano state contattate per accedere a versioni \inglese{trial} dei programmi.

In relazione al numero di componenti del \inglese{team}, purtroppo si sono verificati dei problemi di natura organizzativa in quanto alcuni membri del gruppo non hanno potuto assicurare la loro piena disponibilità. Il \inglese{team} si è perciò ridotto a due soli componenti. In fondo, come detto da Lei stesso:
\begin{quotation}\footnotesize
  <<non è così importante il numero (\dots) quello che conta è la sinergia>>
\end{quotation}

\closing{Distinti saluti,}

%\cc{Other destination}
%\ps{PS: PostScriptum}
%\encl{Enclosures}

\end{letter}
\end{document}
