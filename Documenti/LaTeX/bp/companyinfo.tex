\section{Informazioni sulla società}\label{sec:whoweare}

\subsection{L'idea di impresa}
\team si configura come una società di consulenza aziendale altamente specializzata in soluzioni basate sul paradigma BPM\@. Il \inglese{focus} della sua attività consiste nell'individuare e integrare le soluzioni di qualità esistenti -- sulla scorta di una profonda conoscenza delle alternative disponibili sul mercato sia nazionale che internazionale e con uno speciale interesse nei confronti delle innovazioni più \inglese{cutting edge} del momento -- integrandole con  soluzioni sviluppate \inglese{ad hoc} per i problemi particolari al fine di adattarsi con una precisione `sartoriale' alle esigenze e alle strategie dei singoli clienti.

Le soluzioni proposte da \team rendono possibile innovare il proprio modello di \bsn monitorando in tempo reale lo svolgimento dei processi aziendali al fine di adottare misure correttive \inglese{in itinere} e, grazie alla consapevolezza maturata in tale sede, risolvere in maniera permanente le criticità nell'ottica del miglioramento continuo e del BPR (\inglese{Business Process Reengineering}).

\team nasce nel 2010 con una missione fortemente orientata all'innovazione, allo scopo di fornire soluzioni \sw di alta qualità e a costi `ragionevoli' che mettano i propri clienti nelle condizioni di ottimizzare l'utilizzo delle risorse e migliorare in maniera sostenibile la propria efficienza e la propria competitività in un periodo particolarmente critico per l'economia come quello attuale.

\subsection{Assetto organizzativo}

\subsubsection{Organigramma}
Il \inglese{team} di \team si compone di due esperti di \bsn \inglese{analysis} e organizzazione aziendale, uno specialista di \inglese{project management} e tre ingegneri del \sw come evidenziato nel diagramma riportato in \figurename~\ref{fig:organigram}.

L'ambiente di collaborazione multidisciplinare che scaturisce da una simile combinazione di esperti con \inglese{background} afferenti a discipline profondamente differenti ma che nel corso del tempo hanno evidenziato importanti segnali di convergenza permette all'impresa di creare soluzioni apposite da integrare con quelle esistenti per conseguire gli obiettivi primari di pianificare, monitorare e automatizzare la gestione dei processi aziendali migliorandone l'efficienza ma senza sacrificare le spinte all'innovazione e mantenendo un basso profilo di costi.

\begin{figure}[h]
\centering
\begin{tikzpicture}
\tikzstyle{every node} = [font = \sffamily\bfseries]

\node[drop shadow,fill=white,inner sep=0pt] (head) at (6, 10) {
  \begin{tabular}{|c|}
  \hline
  Titolare\\
  \hline  
  \end{tabular}
};

\node[drop shadow,fill=white,inner sep=0pt] (pm) at (4.5, 9) {
  \begin{tabular}{|c|}
  \hline
  PM\\
  \hline  
  \end{tabular}
};

\node[drop shadow,fill=white,inner sep=0pt] (admin) at (2, 9) {
  \begin{tabular}{|c|}
  \hline
  Amministratore\\
  \hline  
  \end{tabular}
};

\node[drop shadow,fill=white,inner sep=0pt] (mktg) at (7, 9) {
  \begin{tabular}{|c|}
  \hline
  Resp. MKTG\\
  \hline  
  \end{tabular}
};

\node[drop shadow, fill=white, inner sep=0pt] (qa) at (10, 9) {
  \begin{tabular}{|c|}
  \hline
  Resp. QA\\
  \hline  
  \end{tabular}
};

\node[drop shadow,fill=white,inner sep=0pt] (anal) at (4, 8) {
  \begin{tabular}{|c|}
  \hline
  Analista\\
  \hline  
  \end{tabular}
};

\node[drop shadow,fill=white,inner sep=0pt] (prog) at (8, 8) {
  \begin{tabular}{|c|}
  \hline
  Programmatore\\
  \hline  
  \end{tabular}
};

% linea di primo livello
\draw (admin.north) -- ++(0, .2) -- ++(8, 0) -- (qa.north);
\draw (pm.north) -- ++(0, .2);
\draw (mktg.north) -- ++(0, .2);
\draw (head.south) -- ++(0, -.35);

% linea di secondo livello
\draw (anal.north) -- ++(0, .2) -- ++(4, 0) -- (prog.north);
\draw (pm.south) -- ++(0, -.35);

\end{tikzpicture}
\caption{Organigramma che illustra la natura del personale del fornitore.}
\label{fig:organigram}
\end{figure}

\subsubsection{Strategia di \inglese{management}}
\team adotta internamente un \emph{modello integrativo} fortemente orientato al \inglese{commitment} che vede coinvolte al contempo lo sviluppo delle risorse interne, considerate come \inglese{asset} aziendali meritevoli di investimento in formazione continua piuttosto che come mere variabili di costo, e il controllo relativo alla \inglese{compliance} dei dipendenti rispetto agli standard di processo mirando alla costituzione di un ciclo virtuoso in cui le relazioni sociali e la fiducia reciproca sono incoraggiati nel rispetto degli obiettivi e senza perdere di vista il \inglese{focus} sui risultati aziendali.

Le competenze maturate come analista di organizzazione aziendale e solida esperienza e nella gestione delle risorse umane del titolare consentono di mantenere sotto costante controllo l'operato dei dipendenti al fine di garantire il pieno rispetto della pianificazione e la qualità del risultato finale.

\subsection{Le risorse disponibili}
% tecnologia
% brevetti
% prodotti e servizi (make or buy)

\subsection{Traction}
% quali sono i prodotti realizzati? sono in vendita (se no indicare il TTM)
% successi della storia aziendale
% modello business e quanto verrà a costare (o vengono a costare progetti simili)
% obiettivi raggiunti e da raggiungere nel corso del tempo (roadmap)