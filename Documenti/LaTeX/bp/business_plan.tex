%*******************************************************************************
% Macro per il documento corrente
%*******************************************************************************
\newcommand{\sharedPath}{../shared}
\newcommand{\doctitle}{Business Plan}

%*******************************************************************************
% Preambolo
%*******************************************************************************
\documentclass[a4paper,10pt,twoside]{article}

%*******************************************************************************
% Codifica e lingua
%*******************************************************************************
\usepackage[utf8x]{inputenc}
\usepackage[T1]{fontenc}
\usepackage[english,italian]{babel}

%*******************************************************************************
% Qualche macro utile a tutti
%*******************************************************************************
\newcommand{\docRoot}{..}
\newcommand{\inglese}[1]{\foreignlanguage{english}{\textit{#1}}}
\newcommand{\team}{EtaBeta Software\xspace}
\newcommand{\caName}{BPM-1.0\xspace}

%*******************************************************************************
% Figure e immagini
%*******************************************************************************
\usepackage{graphicx}
\graphicspath{{\docRoot/shared/pictures/}}

%*******************************************************************************
% Tabelle
%*******************************************************************************
\usepackage{booktabs}

%*******************************************************************************
% Elenchi puntati personalizzati
%*******************************************************************************
\usepackage{enumitem}

%*************************************************
% Collegamenti intra- e intertestuali
%*************************************************
\usepackage{hyperref}
\hypersetup{%
    colorlinks=false,linktocpage=false,pdfborder={0 0 0},%
    pdfstartpage=1, pdfstartview=FitV,plainpages=false,%
    urlcolor=Black, linkcolor=Black,
    pdfcreator={pdfLaTeX},%
    pdfproducer={pdfLaTeX with hyperref package}%
}

% **************************************************
% Definizione geometria della pagina
% **************************************************
\usepackage[a4paper,head=4cm,top=4.5cm,bottom=3cm,left=3cm,right=3cm,bindingoffset=5mm]{geometry}

%*******************************************************************************
% Altri pacchetti
%*******************************************************************************
\usepackage{xspace} % per spazi condizionali extra
\usepackage{lastpage} % per sapere il numero totale di pagine

% *************************************************
% Intestazioni e piè di pagina personalizzati
% *************************************************
\usepackage{fancyhdr}

% stile normale
\fancypagestyle{normal}{
\fancyhead{}
\fancyhead[LE,RO]{
\sffamily\team
}
\fancyhead[RE,LO]{
\sffamily\leftmark
}
\renewcommand{\headrulewidth}{.4pt}
\cfoot{}
\fancyfoot[RO,LE]{\sffamily
  pag. \thepage{} di \pageref{LastPage}}
\fancyfoot[RE,LO]{\sffamily\doctitle}
\renewcommand{\footrulewidth}{.4pt}
}

% stile per gli indici
\fancypagestyle{toc}{
\fancyhead{}
\fancyhead[LE,RO]{
\sffamily\team
}
\fancyhead[RE,LO]{
\sffamily\caName
}
\renewcommand{\headrulewidth}{.4pt}
\cfoot{}
\fancyfoot[RO,LE]{\sffamily\thepage{}}
\fancyfoot[RE,LO]{\sffamily\doctitle{}}
\renewcommand{\footrulewidth}{.4pt}
}

\pagestyle{fancy}
\renewcommand{\sectionmark}[1]{\markboth{#1}{#1}}


%*******************************************************************************
% Inizio documento
%*******************************************************************************
\begin{document}

\pagestyle{empty}
\begin{center}

{\sffamily
Sviluppo e Gestione Progetti\\
a.a. 2012--2013
}

\vskip 1.5cm

\includegraphics[width=\textwidth]{logo}

\medskip
{\Huge\sffamily\bfseries
\team
}

\vskip 1.5cm

% titolo del progetto
{\Large\sffamily\bfseries
\caName
}

\vskip 1cm

% titolo del documento
\hrule
\vskip 10pt
{\Huge\scshape
\doctitle
}
\vskip 10pt
\hrule

\end{center}

\clearpage

\tableofcontents{\thispagestyle{toc}}

\clearpage

\pagestyle{normal}
\pagenumbering{arabic}


% 1) NDA + DISCLAIMER
% 2) Executive summary
% 3) Descrizione generale della nostra azienda  (~3 pagine)
%   3.1) chi siamo
%   3.2) mission/vision
% 4) Mercato di riferimento: 
%   4.1) analisi della domanda (target di riferimento)
%   4.2) analisi del settore (modello delle 5 forze di Porter), SWOT
%   4.3) descrizione del problema + vantaggi competitivi che derivano dalla soluzione
% 5) Descrizione generale del progetto di sviluppo
%   5.1) descrizione di come è articolata soluzione
%       5.2.1) BPM per ottimizzare processi aziendali - ASPETTARE A SCRIVERLO!
%       5.2.2) corso su come utilizzare software BPM
%       5.2.3) acquisizione figure nuove, ad esempio un analista per capire necessità dei clienti
%       5.2.4) corsi di formazione in base alle carenze rilevate (ad esempio web design + accessibilità)
%        5.2.5) corso di Quality Assurance
%   5.2) risorse richieste per la realizzazione del progetto di sviluppo
%       5.3.1) tempistiche
%       5.3.2) risorse umane coinvolte
%       5.3.2) costi
%       5.3.3) prospettive di guadagno negli anni
\chapter{Note preliminari}

\section{Non-Disclosure Agreement}
Nel ricevere questo documento, vi impegnate a mantenere e garantire la massima riservatezza sulle informazioni ivi contenute, e su quelle di cui verrete a conoscenza, anche solo verbalmente, nel corso di eventuali ulteriori indagini e/o incontri, nonché a restituire immediatamente, su richiesta di \customer, tutto il materiale ricevuto senza trattenere alcuna copia.

Questo documento non dovrà essere fotocopiato, riprodotto o distribuito, per intero o in parte, né citato in documenti ufficiali, senza il preventivo consenso scritto di \customer.

\section{Disclaimer}
Il presente \inglese{business plan} è stato redatto secondo ipotesi, dati e indicazioni formulate e fornite da \customer, alla luce delle informazioni note, della situazione in essere e di quanto poteva essere ragionevolmente supposto, al momento della sua stesura.

Si precisa che, in conformità con l'incarico ricevuto, tali informazioni sono state assunte dai materiali redattori acriticamente, ovvero senza svolgere alcun controllo in merito alla correttezza, completezza e validazione dei dati e informazioni ricevute.

\chapter{Executive summary}\label{sec:summary}
% spiegare perché si dovrebbe investire nel progetto, qual è il suo scopo e perché è idealmente "unico"
% evitare slogan

\chapter{Informazioni sulla società}\label{sec:whoweare}

\section{Chi siamo}
\customer è un'azienda che opera nel settore della comunicazione e del marketing sia on-line che off-line. 
L'organizzazione si occupa in particolare di attività di \inglese{copywriter}, \inglese{mktg} tradizionale, grafica, e negli ultimi anni ha aperto le porte anche nei settori della comunicazione visiva e del \mktg on-line e \inglese{social}. In particolare, per quest'ultimo, le prospettive di mercato sono ottime e \customer prevede di aumentare gli investimenti in tale campo.

\customer è nata alla fine degli anni novanta come società di \inglese{marketing} tradizionale. Negli anni ha saputo rinnovarsi e rimanere al passo con un mercato sempre più dinamico e questa è stata la sua arma vincente.

Oggi, pur essendo un'azienda con un numero limitato di risorse, fattura circa  200.000,00 \text{\euro}.

\section{La nostra Mission}

\textit{``Nel mondo l'abito fa il monaco: il nostro obiettivo è farvi l'abito, il monaco lo dovete mettere voi!''}
%Sono riuscita a fuffare solo questo.... idee migliori?!


\chapter{Mercato di riferimento}
\section{Analisi della domanda}\label{sec:domanda}

Oggi il mercato si presenta molto dinamico, incline a continui cambiamenti e pieno di  \inglese{competitor}, in qualsiasi settore si operi. Il \mktg , in qualsiasi forma sia fatto, permette alle aziende di emergere rispetto ad altre e di reperire clienti, che costituiscono la fonte del guadagno di ogni organizzazione.

È  proprio per questo che, il \inglese{target} dei clienti delle aziende che operano nel settore del \mktg, come \customer , è molto vario e ampio. Qualsiasi azienda, di grandi o piccole dimensioni, può detenere interessi nel richiedere un intervento di consulenza \mktg con il fine di ampliare il suo \textit{portfolio} clienti.

Oggi, le aziende, puntano soprattutto al potenziamento del \mktg \inglese{online}. Infatti, negli ultimi anni, la crescita esponenziale del numero di utenti connessi alla rete ha moltiplicato le probabilità di reperire nuovi clienti e mantenere i contatti con quelli esistenti, proprio tramite \inglese{Internet}.

Molti e diversificati sono quindi i soggetti che in particolare nel Veneto necessitano del \inglese{know-how} e delle competenze trasversali che \customer mette a disposizione per il \inglese{digital} \mktg (in ambito \inglese{web} e \inglese{social}) e per l'ottimizzazione dei siti e del posizionamento nei motori di ricerca di questi ultimi.

\customer si colloca funzionalmente all'interno del tessuto piccolo e medio-industriale dell'area nordorientale del paese, con lo scopo di fungere idealmente da tramite dalla piccola realtà territoriale al mercato globale sia esso interno al paese che, grazie alla collaborazione con fornitori di servizi linguistici, esterno e internazionale.

\customer offre ai propri clienti un'ampia gamma di servizi che spaziano dall'elaborazione di strategie di \mktg al posizionamento sui motori di ricerca, dalla pubblicità \inglese{online} agli studi di \inglese{web analytics}, dalla creazione di contenuti all'internazionalizzazione di contenuti esistenti, fino agli aspetti \inglese{social} con la creazione di \inglese{blog} aziendali all'utilizzo di \inglese{Social Media Platforms} (\swname{Facebook}, \swname{LinkedIn}, \swname{Google+}, \swname{YouTube}).

%TODO da dove vengono questi dati?
Molte aziende decidono tuttavia di affidare le attività di \mktg a personale interno. Da uno studio del mercato nazionale, emerge che circa il 19\% delle aziende preferisce affidare il compito ad organizzazioni esterne specializzate.

Tale dato, riduce di molto il \inglese{target} effettivo di clienti che si rivolge ad aziende specializzate nel settore del \mktg.
Tuttavia, dallo stesso studio, emerge che circa il 97\% delle aziende utilizza il web come strumento di \mktg. È  proprio per questo motivo che, specializzandosi nel \mktg \inglese{on-line} e \inglese{social} le aziende operanti nel settore hanno maggior possibilità di acquisire tutti quei clienti che, pur affidando le attività di \mktg a settori interni,  mancano di preparazione ed esperienza nel campo web.

L'area del Nordest rappresenta indubbiamente un settore particolarmente critico da questo punto di vista, dal momento che le PMI collocate in quest'area geografica non sempre sono in grado di sfruttare appieno le potenzialità offerte dalla rete per la commercializzazione dei prodotti e il lancio delle promozioni \cite[\itshape{}pag. 6 e succ.]{bassi:pmi}.

In particolare, lo studio evidenzia una scarsa efficacia nella comunicazione con il cliente, la diffusa incapacità di cogliere le nuovi occasioni di \bsn, i pochi collegamenti ai \inglese{social}, l'esiguità dei contenuti messi a disposizione in rete e una tendenza alla diffidenza nei confronti dell'\inglese{e-commerce}.

Uno scenario così configurato contiene, seppur allo stadio embrionale, numerose possibilità di espansione per il futuro, specialmente tenendo conto che l'\inglese{e-business} può essere volano per la ripresa dell'economia in un momento critico come quello attuale. È per questo motivo che \customer si andata specializzando sempre più nel \mktg digitale nel corso dei suoi vent'anni di presenza sul mercato.

\section{Analisi del settore}
% dimensioni e andamento del settore
% barriere all'entrata
% influenze sul settore di cambiamenti macroeconomici
% profittabilità e posizione finanziaria del settore
% ruolo dell'innovazione tecnologica
% influenza di regolamenti/normative
% vantaggio competitivo
% analisi della domanda: trend di crescita/sviluppo della richiesta
% consierare il modello delle 5 forze di Porter (sistema competitivo allargato)

%siti di concorrenti
% http://www.nordestinnovazione.it
% http://www.webranking.it/
% http://www.kreatif-multimedia.com/it
% (modello delle 5 forze di Porter), SWOT

\section{Descrizione del problema}
La \customer , pur essendo un'ottima azienda che riesce, nonostante la crisi attuale del mercato, a fatturare  200.000,00 \text{\euro} all'anno, presenta diversi problemi relativi allo svolgimento dei processi.
I processi costituiscono il cuore di un'azienda e quindi, per mantenere l'azienda un \inglese{leader} del mercato in cui opera, è fondamentale avere una buona organizzazione dei processi.
I problemi che si sono rilevati all'interno dei processi di \customer si riferiscono in particolare alle seguenti fasi nel ciclo di realizzazione di un progetto di consulenza:
\begin{itemize}
	\item Analisi delle necessità;
	\item Predisposizione offerta;
	\item Negoziazione e sottoscrizione offerta.
\end{itemize}

L'organizzazione ha rilevato la presenza di diverse problematiche relative alla fase di analisi di necessità riguardanti le richieste di un cliente. Vi sono stati infatti, diversi casi in cui, \customer  ha dovuto eseguire delle modifiche ai progetti di consulenza a sue spese, in quanto, non avendo compreso a pieno le necessità del cliente, ha  realizzato in maniera inadeguata la consulenza. 

Sono inoltre stati incontrati dei problemi anche nella stesura delle proposte di consulenza. La forma con cui si predispone un'offerta è molto importante. Pur essendo il contenuto di una proposta la fonte primaria di accettazione da parte del cliente, la forma con cui si espone il contenuto influenza il verdetto finale.
Da questo derivano inoltre problematiche relative alla negoziazione e sottoscrizione di un'offerta. La \customer , spesso, non è in grado di ``condizionare'' il cliente e portare la negoziazione ad una convergenza vantaggiosa per entrambe le parti.

Il piano di sviluppo avrà quindi l'obiettivo di risolvere tali problematiche, aumentare l'efficienza e la produttività in modo tale da permettere ad \customer di ricoprire un ruolo da protagonista nello scenario di mercato degli anni a venire.


\chapter{Descrizione generale del progetto di sviluppo}

\section{Descrizione di come è articolata soluzione}
\subsubsection{BPM per ottimizzare processi aziendali}

\subsubsection{Corso su come utilizzare software BPM}

\subsubsection{Acquisizione figure nuove}
%, ad esempio un analista per capire necessità dei clienti

\subsubsection{Corsi di formazione in base alle carenze rilevate} 
%(ad esempio web design + accessibilità)

\subsubsection{Corso di Quality Assurance}
  
\section{Risorse richieste per la realizzazione del progetto di sviluppo}
\subsection{Tempistiche}
La \customer si pone l'obbiettivo di portare a termine il piano di sviluppo in 2 anni. l'azienda si prefissa quindi di realizzare l'interno piano entro dicembre 2015.       
	
La \customer prevede di realizzare le attività secondo il seguente ordine:
\begin{enumerate}
\item adozione di \inglese{software} BPM
\item corso di formazione del personale per l'utilizzo del software BPM
\item acquisizione nuove figure
\item corsi di formazione 
\item corso di Quality Assurance
\end{enumerate}   
La seguente tabella espone in dettaglio la tempistica con la quale l'azienda intende realizzare i punti previsti dal piano di sviluppo.
     
\begin{table}[H]
\centering
\begin{tabular}{|p{.50\textwidth}|c|c|c|}
\hline

\textbf{ Attività} & \textbf{Inizio} & \textbf{Durata}\\
\hline
 adozione \inglese{software} BPM & giugno 2103 & permanente \\
 corso di formazione del personale per l'utilizzo del software BPM &  settembre 2013 & 40 ore \\
 acquisizione nuove figure & novembre 2013 & permanente  \\
 corsi di formazione & gennaio 2014 & varia \footnote {il tempo di durata del corso varia a seconda della tipologia del corso corso e delle risorse coinvolte}\\
 corso di Quality Assurance&  aprile 2014 & 300 ore \\

\hline

\end{tabular}
\caption{Tempi di realizzazione sviluppo}\label{tab:tempi}
\end{table}
 
 Si noti che le tempistiche specificate sono solo di carattere indicativo in quanto il reale svolgimento nei tempi previsti dipende, in gran parte, da fattori esterni all'organizzazione. Inoltre, non vi è un'assoluta sicurezza temporale relativa all'effettivo reperimento dei fondi stanziati per il piano di sviluppo.

\subsection{Risorse umane coinvolte}
\subsection{Costi}
\subsection{ Prospettive di guadagno negli anni}


\begin{thebibliography}{90}
  \bibitem[Bassi, 2012]{bassi:pmi} Bassi, F., \textit{PMI bocciate alla prova del Web: la conquista di Internet è ancora lontana}, 2012, \url{http://fondazionecomunica.org/UserFiles/files/Rapporto_RicercaCompleta%20per%20sito(1).pdf} [data di consultazione 13/07/2013]
\end{thebibliography}

\end{document}
