%*******************************************************************************
% Macro per il documento corrente
%*******************************************************************************
\newcommand{\sharedPath}{../shared}
\newcommand{\doctitle}{Business Plan}


%*******************************************************************************
% Preambolo
%*******************************************************************************
\documentclass[a4paper,10pt,twoside]{article}

%*******************************************************************************
% Codifica e lingua
%*******************************************************************************
\usepackage[utf8x]{inputenc}
\usepackage[T1]{fontenc}
\usepackage[english,italian]{babel}

%*******************************************************************************
% Qualche macro utile a tutti
%*******************************************************************************
\newcommand{\docRoot}{..}
\newcommand{\inglese}[1]{\foreignlanguage{english}{\textit{#1}}}
\newcommand{\team}{EtaBeta Software\xspace}
\newcommand{\caName}{BPM-1.0\xspace}

%*******************************************************************************
% Figure e immagini
%*******************************************************************************
\usepackage{graphicx}
\graphicspath{{\docRoot/shared/pictures/}}

%*******************************************************************************
% Tabelle
%*******************************************************************************
\usepackage{booktabs}

%*******************************************************************************
% Elenchi puntati personalizzati
%*******************************************************************************
\usepackage{enumitem}

%*************************************************
% Collegamenti intra- e intertestuali
%*************************************************
\usepackage{hyperref}
\hypersetup{%
    colorlinks=false,linktocpage=false,pdfborder={0 0 0},%
    pdfstartpage=1, pdfstartview=FitV,plainpages=false,%
    urlcolor=Black, linkcolor=Black,
    pdfcreator={pdfLaTeX},%
    pdfproducer={pdfLaTeX with hyperref package}%
}

% **************************************************
% Definizione geometria della pagina
% **************************************************
\usepackage[a4paper,head=4cm,top=4.5cm,bottom=3cm,left=3cm,right=3cm,bindingoffset=5mm]{geometry}

%*******************************************************************************
% Altri pacchetti
%*******************************************************************************
\usepackage{xspace} % per spazi condizionali extra
\usepackage{lastpage} % per sapere il numero totale di pagine

% *************************************************
% Intestazioni e piè di pagina personalizzati
% *************************************************
\usepackage{fancyhdr}

% stile normale
\fancypagestyle{normal}{
\fancyhead{}
\fancyhead[LE,RO]{
\sffamily\team
}
\fancyhead[RE,LO]{
\sffamily\leftmark
}
\renewcommand{\headrulewidth}{.4pt}
\cfoot{}
\fancyfoot[RO,LE]{\sffamily
  pag. \thepage{} di \pageref{LastPage}}
\fancyfoot[RE,LO]{\sffamily\doctitle}
\renewcommand{\footrulewidth}{.4pt}
}

% stile per gli indici
\fancypagestyle{toc}{
\fancyhead{}
\fancyhead[LE,RO]{
\sffamily\team
}
\fancyhead[RE,LO]{
\sffamily\caName
}
\renewcommand{\headrulewidth}{.4pt}
\cfoot{}
\fancyfoot[RO,LE]{\sffamily\thepage{}}
\fancyfoot[RE,LO]{\sffamily\doctitle{}}
\renewcommand{\footrulewidth}{.4pt}
}

\pagestyle{fancy}
\renewcommand{\sectionmark}[1]{\markboth{#1}{#1}}


%*******************************************************************************
% Inizio documento
%*******************************************************************************
\begin{document}

\pagestyle{empty}
\begin{center}

{\sffamily
Sviluppo e Gestione Progetti\\
a.a. 2012--2013
}

\vskip 1.5cm

\includegraphics[width=\textwidth]{logo}

\medskip
{\Huge\sffamily\bfseries
\team
}

\vskip 1.5cm

% titolo del progetto
{\Large\sffamily\bfseries
\caName
}

\vskip 1cm

% titolo del documento
\hrule
\vskip 10pt
{\Huge\scshape
\doctitle
}
\vskip 10pt
\hrule

\end{center}

\clearpage

\tableofcontents{\thispagestyle{toc}}

\clearpage

\pagestyle{normal}
\pagenumbering{arabic}


\section{Introduzione}

\subsection{Non-Disclosure Agreement}
Nel ricevere questo documento, vi impegnate a mantenere e garantire la massima riservatezza sulle informazioni ivi contenute, e su quelle di cui verrete a conoscenza, anche solo verbalmente, nel corso di eventuali ulteriori indagini e/o incontri, nonché a restituire immediatamente, su richiesta di \team, tutto il materiale ricevuto senza trattenere alcuna copia.

Questo documento non dovrà essere fotocopiato, riprodotto o distribuito, per intero o in parte, né citato in documenti ufficiali, senza il preventivo consenso scritto di \team.

\subsection{Disclaimer}
Il presente \inglese{business plan} è stato redatto secondo ipotesi, dati e indicazioni formulate e fornite da \customer, alla luce delle informazioni note, della situazione in essere e di quanto poteva essere ragionevolmente supposto, al momento della sua stesura.

Si precisa che, in conformità con l'incarico ricevuto, tali informazioni sono state assunte dai materiali redattori acriticamente, ovvero senza svolgere alcun controllo in merito alla correttezza, completezza e validazione dei dati e informazioni ricevute.


\subsection{Scopo del documento}
Il presente documento si propone di fornire informazioni su \team, descrivendo quali sono le abilità e le competenze che intende mettere in campo per la realizzazione del progetto commissionato da \customer delineando una \inglese{roadmap} di sviluppo attraverso cui raggiungere gli obiettivi di progetto (sezione~\ref{sec:whoweare}).

In seguito, verranno presentati con maggiori dettagli gli obiettivi del progetto, i problemi che la soluzione proposta si propone di affrontare e le funzionalità che quest'ultima dovrà soddisfare tramite l'illustrazione degli scenari tipici di utilizzo (sezione \ref{sec:whattheproblemis}).
 
Infine, sarà illustrata la strategia di uscita, descrivendo con maggiori dettagli perché la soluzione proposta risolve il problema, le strategie di \mktg e quali possono essere gli utilizzi futuri e i potenziali acquirenti esterni in una prospettiva di analisi finanziaria (sezione \ref{sec:exitstrategy}).

\subsection{Executive summary}
%TODO questo va scritto alla fine, non si può fuffare ora!
% forse deve collassare con la sezione precedente

% spiegare perché si dovrebbe investire nel progetto, qual è il suo scopo e perché è idealmente "unico"
% no slogan
% riassumere i punti che saranno trattati nel seguito del documento:
%   - società
%   - prodotto
%   - mercato
%   - area finanziaria
%   - management e team di sviluppo

\clearpage

\section{Informazioni sulla società}\label{sec:whoweare}

\subsection{L'idea di impresa}
\team si configura come una società di consulenza aziendale altamente specializzata in soluzioni basate sul paradigma BPM\@. Il \inglese{focus} della sua attività consiste nell'individuare e integrare le soluzioni di qualità esistenti -- sulla scorta di una profonda conoscenza delle alternative disponibili sul mercato sia nazionale che internazionale e con uno speciale interesse nei confronti delle innovazioni più \inglese{cutting edge} del momento -- integrandole con  soluzioni sviluppate \inglese{ad hoc} per i problemi particolari al fine di adattarsi con una precisione `sartoriale' alle esigenze e alle strategie dei singoli clienti.

Le soluzioni proposte da \team rendono possibile innovare il proprio modello di \bsn monitorando in tempo reale lo svolgimento dei processi aziendali al fine di adottare misure correttive \inglese{in itinere} e, grazie alla consapevolezza maturata in tale sede, risolvere in maniera permanente le criticità nell'ottica del miglioramento continuo.

\team nasce nel 2010 con una missione fortemente orientata all'innovazione, allo scopo di fornire soluzioni \sw di alta qualità e a costi `ragionevoli' che mettano i propri clienti nelle condizioni di ottimizzare l'utilizzo delle risorse e migliorare in maniera sostenibile la propria efficienza in un periodo particolarmente critico per l'economia.

\subsection{Assetto organizzativo}

\subsection{Organigramma}
%TODO da fare
% - analista
% - titolare
% - amministratore
% - programmatore
% - pm
% - responsabile mktg

\begin{figure}[h]
\centering
\begin{tikzpicture}
\tikzstyle{every node} = [font = \sffamily\bfseries]

\node[drop shadow,fill=white,inner sep=0pt] (titolare) at (6, 10) {
  \begin{tabular}{|c|}
  \hline
  Titolare\\
  \hline  
  \end{tabular}
};

\node[drop shadow,fill=white,inner sep=0pt] (pm) at (6, 9) {
  \begin{tabular}{|c|}
  \hline
  Project Manager\\
  \hline  
  \end{tabular}
};

\node[drop shadow,fill=white,inner sep=0pt] (admin) at (2, 9) {
  \begin{tabular}{|c|}
  \hline
  Amministratore\\
  \hline  
  \end{tabular}
};

\node[drop shadow,fill=white,inner sep=0pt] (mktg) at (10, 9) {
  \begin{tabular}{|c|}
  \hline
  Responsabile MKTG\\
  \hline  
  \end{tabular}
};

\node[drop shadow,fill=white,inner sep=0pt] (anal) at (4, 8) {
  \begin{tabular}{|c|}
  \hline
  Analista\\
  \hline  
  \end{tabular}
};

\node[drop shadow,fill=white,inner sep=0pt] (prog) at (8, 8) {
  \begin{tabular}{|c|}
  \hline
  Programmatore\\
  \hline  
  \end{tabular}
};

% linea di primo livello
\draw (admin.north) -- ++(0, .2) -- ++(8, 0) -- (mktg.north);
\draw (pm.north) -- ++(0, .55);

% linea di secondo livello
\draw (anal.north) -- ++(0, .2) -- ++(4, 0) -- (prog.north);
\draw (pm.south) -- ++(0, -.35);

\end{tikzpicture}
\end{figure}

\subsection{Strategia di \inglese{management}}
% evidenziare il management solido e le competenze dei dirigenti: il progetto non dovrà fallire o essere consegnato in ritardo

\subsection{Le risorse disponibili}
Il \inglese{team} di \team è composto in parte da esperti di \bsn \inglese{analysis} e organizzazione aziendale e in parte da ingegneri del \sw in grado di creare soluzioni apposite da integrare con quelle esistenti per conseguire gli obiettivi prefissati e automatizzare la gestione dei processi aziendali migliorandone l'efficienza.
% tecnologia
% brevetti
% prodotti e servizi (make or buy)

\subsection{Traction}
% quali sono i prodotti realizzati? sono in vendita (se no indicare il TTM)
% successi della storia aziendale
% modello business e quanto verrà a costare (o vengono a costare progetti simili)
% obiettivi raggiunti e da raggiungere nel corso del tempo (roadmap)

\clearpage

\section{Il mercato di riferimento}\label{sec:whattheproblemis}
%TODO Citare sempre fonti a supporto delle affermazioni
\subsection{Analisi di mercato}
% descrivere mercato target e clientela attuale
% dimensioni del target di mercato (in prospettiva diacronica)
% segmentazione del mercato target
% come avviene la distribuzione

\subsection{Analisi di settore}
% dimensioni e andamento del settore
% barriere all'entrata
% influenze sul settore di cambiamenti macroeconomici
% profittabilità e posizione finanziaria del settore
% ruolo dell'innovazione tecnologica
% influenza di regolamenti/normative
% vantaggio competitivo
% analisi della domanda: trend di crescita/sviluppo della richiesta
% consierare il modello delle 5 forze di Porter (sistema competitivo allargato)

\subsection{Descrizione del progetto}
% in linea di massima: contenuti, obiettivi, tempi
% - problem breakdown
% - SWOT e cazzivari
% - UC
% - obiettivi e milestones

\clearpage

\section{Strategie di \mktg}\label{sec:exitstrategy}
% tecniche di mktg (eventualmente riferendo progetti già esistenti)
% scelte operative e scelte produttive
% piano di mktg
% piano commerciale
% segmento del mercato target per area geografica e tipo di potenziale acquirente
% il target market è soddisfatto dalla concorrenza?
% prezzo proposto / garanzie sui prodotti --> noi scegliamo di adottare una strategica di leadership di costo riutilizzando il più possibile soluzioni esistenti per mantenere la soluzione proposta su costi sostenibili e `contenuti' rispetto alla concorrenza
% immagine da trasmettere e credibilità aziendale
% investimenti in PR
% canali distributivi e modalità di vendita + assistenza post-vendita

\clearpage

\section{Analisi finanziaria}
% prospettive di vendite e guadagni rispetto alle spese sostenute (almeno su 3 anni)
% capitale come grafico in prospettiva

\end{document}