%*******************************************************************************
% Macro per il documento corrente
%*******************************************************************************
\newcommand{\sharedPath}{../shared}
\newcommand{\doctitle}{Business Plan}

%*******************************************************************************
% Preambolo
%*******************************************************************************
\documentclass[a4paper,10pt,twoside]{article}

%*******************************************************************************
% Codifica e lingua
%*******************************************************************************
\usepackage[utf8x]{inputenc}
\usepackage[T1]{fontenc}
\usepackage[english,italian]{babel}

%*******************************************************************************
% Qualche macro utile a tutti
%*******************************************************************************
\newcommand{\docRoot}{..}
\newcommand{\inglese}[1]{\foreignlanguage{english}{\textit{#1}}}
\newcommand{\team}{EtaBeta Software\xspace}
\newcommand{\caName}{BPM-1.0\xspace}

%*******************************************************************************
% Figure e immagini
%*******************************************************************************
\usepackage{graphicx}
\graphicspath{{\docRoot/shared/pictures/}}

%*******************************************************************************
% Tabelle
%*******************************************************************************
\usepackage{booktabs}

%*******************************************************************************
% Elenchi puntati personalizzati
%*******************************************************************************
\usepackage{enumitem}

%*************************************************
% Collegamenti intra- e intertestuali
%*************************************************
\usepackage{hyperref}
\hypersetup{%
    colorlinks=false,linktocpage=false,pdfborder={0 0 0},%
    pdfstartpage=1, pdfstartview=FitV,plainpages=false,%
    urlcolor=Black, linkcolor=Black,
    pdfcreator={pdfLaTeX},%
    pdfproducer={pdfLaTeX with hyperref package}%
}

% **************************************************
% Definizione geometria della pagina
% **************************************************
\usepackage[a4paper,head=4cm,top=4.5cm,bottom=3cm,left=3cm,right=3cm,bindingoffset=5mm]{geometry}

%*******************************************************************************
% Altri pacchetti
%*******************************************************************************
\usepackage{xspace} % per spazi condizionali extra
\usepackage{lastpage} % per sapere il numero totale di pagine

% *************************************************
% Intestazioni e piè di pagina personalizzati
% *************************************************
\usepackage{fancyhdr}

% stile normale
\fancypagestyle{normal}{
\fancyhead{}
\fancyhead[LE,RO]{
\sffamily\team
}
\fancyhead[RE,LO]{
\sffamily\leftmark
}
\renewcommand{\headrulewidth}{.4pt}
\cfoot{}
\fancyfoot[RO,LE]{\sffamily
  pag. \thepage{} di \pageref{LastPage}}
\fancyfoot[RE,LO]{\sffamily\doctitle}
\renewcommand{\footrulewidth}{.4pt}
}

% stile per gli indici
\fancypagestyle{toc}{
\fancyhead{}
\fancyhead[LE,RO]{
\sffamily\team
}
\fancyhead[RE,LO]{
\sffamily\caName
}
\renewcommand{\headrulewidth}{.4pt}
\cfoot{}
\fancyfoot[RO,LE]{\sffamily\thepage{}}
\fancyfoot[RE,LO]{\sffamily\doctitle{}}
\renewcommand{\footrulewidth}{.4pt}
}

\pagestyle{fancy}
\renewcommand{\sectionmark}[1]{\markboth{#1}{#1}}


%*******************************************************************************
% Inizio documento
%*******************************************************************************
\begin{document}

\pagestyle{empty}
\begin{center}

{\sffamily
Sviluppo e Gestione Progetti\\
a.a. 2012--2013
}

\vskip 1.5cm

\includegraphics[width=\textwidth]{logo}

\medskip
{\Huge\sffamily\bfseries
\team
}

\vskip 1.5cm

% titolo del progetto
{\Large\sffamily\bfseries
\caName
}

\vskip 1cm

% titolo del documento
\hrule
\vskip 10pt
{\Huge\scshape
\doctitle
}
\vskip 10pt
\hrule

\end{center}

\clearpage

\tableofcontents{\thispagestyle{toc}}

\clearpage

\pagestyle{normal}
\pagenumbering{arabic}


% 1) NDA + DISCLAIMER
% 2) Executive summary
% 3) Descrizione generale della nostra azienda  (~3 pagine)
%   3.1) chi siamo
%   3.2) mission/vision
% 4) Mercato di riferimento: 
%   4.1) analisi della domanda (target di riferimento)
%   4.2) analisi del settore (modello delle 5 forze di Porter), SWOT
%   4.3) descrizione del problema + vantaggi competitivi che derivano dalla soluzione
% 5) Descrizione generale del progetto di sviluppo
%   5.1) descrizione di come è articolata soluzione
%       5.2.1) BPM per ottimizzare processi aziendali - ASPETTARE A SCRIVERLO!
%       5.2.2) corso su come utilizzare software BPM
%       5.2.3) acquisizione figure nuove, ad esempio un analista per capire necessità dei clienti
%       5.2.4) corsi di formazione in base alle carenze rilevate (ad esempio web design + accessibilità)
%        5.2.5) corso di Quality Assurance
%   5.2) risorse richieste per la realizzazione del progetto di sviluppo
%       5.3.1) tempistiche
%       5.3.2) risorse umane coinvolte
%       5.3.2) costi
%       5.3.3) prospettive di guadagno negli anni
\section{Note preliminari}

\subsection{Non-Disclosure Agreement}
Nel ricevere questo documento, vi impegnate a mantenere e garantire la massima riservatezza sulle informazioni ivi contenute, e su quelle di cui verrete a conoscenza, anche solo verbalmente, nel corso di eventuali ulteriori indagini e/o incontri, nonché a restituire immediatamente, su richiesta di \customer, tutto il materiale ricevuto senza trattenere alcuna copia.

Questo documento non dovrà essere fotocopiato, riprodotto o distribuito, per intero o in parte, né citato in documenti ufficiali, senza il preventivo consenso scritto di \customer.

\subsection{Disclaimer}
Il presente \inglese{business plan} è stato redatto secondo ipotesi, dati e indicazioni formulate e fornite da \customer, alla luce delle informazioni note, della situazione in essere e di quanto poteva essere ragionevolmente supposto, al momento della sua stesura.

Si precisa che, in conformità con l'incarico ricevuto, tali informazioni sono state assunte dai materiali redattori acriticamente, ovvero senza svolgere alcun controllo in merito alla correttezza, completezza e validazione dei dati e informazioni ricevute.

\clearpage

\section{Executive summary}\label{sec:summary}
% spiegare perché si dovrebbe investire nel progetto, qual è il suo scopo e perché è idealmente "unico"
% evitare slogan

\clearpage

\section{Informazioni sulla società}\label{sec:whoweare}

\subsection{Chi siamo}
\customer è un'azienda che opera nel settore della comunicazione e del marketing sia on-line che off-line. 
L'organizzazione si occupa in particolare di attività di \inglese{copywriter}, \inglese{mktg} tradizionale, grafica, e negli ultimi anni ha aperto le porte anche nei settori della comunicazione visiva e del \mktg on-line e \inglese{social}. In particolare, per quest'ultimo, le prospettive di mercato sono ottime e \customer prevede di aumentare gli investimenti in tale campo.

\customer è nata alla fine degli anni novanta come società di \inglese{marketing} tradizionale. Negli anni ha saputo rinnovarsi e rimanere al passo con un mercato sempre più dinamico e questa è stata la sua arma vincente.

Oggi, pur essendo un'azienda con un numero limitato di risorse, fattura circa  200.000,00 \text{\euro}.

\subsection{La nostra Mission}

\textit{``Nel mondo l'abito fa il monaco: il nostro obiettivo è farvi l'abito, il monaco lo dovete mettere voi!''}
%Sono riuscita a fuffare solo questo.... idee migliori?!






\section{Mercato di riferimento} 
\subsection{Analisi della domanda} %(target di riferimento)
\subsection{Analisi del settore}
% (modello delle 5 forze di Porter), SWOT
\subsection{Descrizione del problema}

La \customer , pur essendo un'ottima azienda che riesce, nonostante la crisi attuale del mercato, a fatturare  200.000,00 \text{\euro} all'anno, presenta diversi problemi relativi allo svolgimento dei processi.
I processi costituiscono il cuore di un'azienda e quindi, per mantenere l'azienda un \inglese{leader} del mercato in cui opera, è fondamentale avere una buona organizzazione dei processi.
I problemi che si sono rilevati all'interno dei processi di \customer si riferiscono in particolare alle seguenti fasi nel ciclo di realizzazione di un progetto di consulenza:
\begin{itemize}
	\item Analisi delle necessità;
	\item Predisposizione offerta;
	\item Negoziazione e sottoscrizione offerta.
\end{itemize}


L'organizzazione ha rilevato la presenza di diverse problematiche relative alla fase di analisi di necessità riguardanti le richieste di un cliente. Vi sono stati infatti, diversi casi in cui, \customer  ha dovuto eseguire delle modifiche ai progetti di consulenza a sue spese, in quanto, non avendo compreso a pieno le necessità del cliente, ha  realizzato in maniera inadeguata la consulenza. 

Sono inoltre stati incontrati dei problemi anche nella stesura delle proposte di consulenza. La forma con cui si predispone un'offerta è molto importante. Pur essendo il contenuto di una proposta la fonte primaria di accettazione da parte del cliente, la forma con cui si espone il contenuto influenza il verdetto finale.
Da questo derivano inoltre problematiche relative alla negoziazione e sottoscrizione di un'offerta. La \customer , spesso, non è in grado di ``condizionare'' il cliente e portare la negoziazione ad una convergenza vantaggiosa per entrambe le parti.




Il piano di sviluppo avrà quindi l'obiettivo di risolvere tali problematiche, aumentare l'efficienza e la produttività in modo tale da permettere ad \customer di ricoprire un ruolo da protagonista nello scenario di mercato degli anni a venire.










%Descrizione del problema
 % vantaggi competitivi che derivano dalla soluzione


\section {Descrizione generale del progetto di sviluppo}
	\subsection{Descrizione di come è articolata soluzione}
    \subsubsection{BPM per ottimizzare processi aziendali}
   \subsubsection{Corso su come utilizzare software BPM}
      \subsubsection{Acquisizione figure nuove}
      %, ad esempio un analista per capire necessità dei clienti
     \subsubsection{Corsi di formazione in base alle carenze rilevate} 
     %(ad esempio web design + accessibilità)
        \subsubsection{Corso di Quality Assurance}
  
   \subsection{Risorse richieste per la realizzazione del progetto di sviluppo}
       \subsubsection{Tempistiche}
       \subsubsection{Risorse umane coinvolte}
      \subsubsection{Costi}
       \subsubsection{ Prospettive di guadagno negli anni}





\clearpage

\section{Analisi di mercato}
%TODO citare fonti a supporto delle affermazioni
\subsection{Analisi della domanda}
% descrivere mercato target e clientela attuale
% dimensioni del target di mercato (in prospettiva diacronica)
% segmentazione del mercato target
% come avviene la distribuzione

\subsection{Analisi di settore}
% dimensioni e andamento del settore
% barriere all'entrata
% influenze sul settore di cambiamenti macroeconomici
% profittabilità e posizione finanziaria del settore
% ruolo dell'innovazione tecnologica
% influenza di regolamenti/normative
% vantaggio competitivo
% analisi della domanda: trend di crescita/sviluppo della richiesta
% consierare il modello delle 5 forze di Porter (sistema competitivo allargato)

L'adozione di una soluzione di BPM ha il vantaggio di comportare un incremento dell'efficienza a seguito dell'automazione delle stesse nonché l'eliminazione degli \inglese{step} intermedi non necessari dovuta all'acquisizione di maggior consapevolezza derivante dalla formalizzazione dei processi aziendali in modelli astratti (situazione corrente `as is').

Implicando la standardizzazione del metodo di lavoro, l'implementazione di un \inglese{workflow management system} come conseguenza del BPM permette l'adozione di strumenti di verifica e di generazione automatica di \inglese{report} sulle attività ordinarie dell'azienda che consentono di monitorare gli indicatori chiave con estrema accuratezza. Si tratta di condizioni imprescindibili per una buona pianificazione e per l'ottimizzazione dei processi (situazione ideale o `to be') e per passare dalla semplice modellazione dei processi aziendali (\bsn \inglese{process modeling}) a una rivisitazione critica degli stessi (\bsn \inglese{process reengineering}).

La gestione della non-linearità dei processi, inoltre, denota la flessibilità di tali sistemi, che sono in grado di soddisfare le istanze di modellazione di situazioni che non sono prevedibili aprioristicamente, per adattarsi ai cicli di vita di sempre più ridotte dimensioni per la gestione degli ordini e a un ambiente competitivo in costante evoluzione.

Infine, tramite una gestione controllata dei processi, è possibile incrementare l'efficienza anche in caso di \inglese{workflow} collaborativi, rendendo più trasparente lo scambio di informazioni fra la totalità dei soggetti coinvolti, la condivisione della conoscenza nonché la drastica riduzione dei tempi di accesso alle basi di conoscenza aziendali, incrementandone al contempo l'utilità e le dimensioni perché il flusso di lavoro diviene tracciato, codificato e ripetibile anche in caso di situazioni potenzialmente complesse in maniera indipendente dalla soggettività/esperienza degli incaricati.

La soluzione BPM da adottare dovrà quindi essere in grado di garantire le seguenti funzionalità minime:
\begin{itemize}
  \item[--] rappresentare con un certo livello di astrazione gli attuali processi aziendali di \customer al fine di permetterne la pianificazione (\bsn \inglese{process modeling}) e l'eventuale modifica;
  \item[--] registrare lo svolgimento del processo secondo gli standard del modello configurati nella fase precedente in forma quanto più `trasparente' per gli utilizzatori integrandosi il più possibile all'interno del normale svolgimento delle attività;
  \item[--] fornire adeguati strumenti di \inglese{report} e diagnostica di difetti nella gestione, circoli che potrebbero causare ritardi o, in generale, fonti di criticità;
  \item[--] mettere a disposizione degli operatori i dati prodotti dagli altri soggetti coinvolti per facilitare la collaborazione e gestire i contenuti documentali e i metadati di processo aumentando in tal modo la base di conoscenza sia implicita che esplicita di \customer.
\end{itemize}

\begin{figure}[H]
  \centering
  \begin{tikzpicture}
\tikzstyle{every node} = [font = \sffamily\footnotesize]


\draw[semithick] (1, 5) circle (.1);
\draw[semithick] (.9, 4.8) -- ++(.1, .1) -- ++(.1, -.1);
\node at (1.1, 4.7) (commerciale) {};
\draw[semithick] (1, 4.9) -- ++(0, -.25);
\draw[semithick] (.9, 4.55) -- ++(.1, .1) -- ++(.1, -.1) node[below]{commerciale};

\node[draw, ellipse] (uc1) at (6, 5.5)  {UC1.1 creazione scheda cliente };
\node[draw, ellipse] (uc2) at (6, 4.8)  {UC1.2 creazione offerta };
\node[draw, ellipse] (uc3) at (6, 4)  {UC1.3 modifica stato offerta };

\draw(commerciale.east) -- (uc1.west)
  (commerciale.east) -- (uc2.west)
  (commerciale.east) -- (uc3.west);

\draw (3, 3.5) rectangle ++(6, 2.8) node[below left] {ERP-1};

\end{tikzpicture}

  \caption{Operazioni svolte dal commerciale (UC1)}
\end{figure}

\begin{figure}[H]
  \centering
  \begin{tikzpicture}
\tikzstyle{every node} = [font = \sffamily\footnotesize]


\draw[semithick] (1, 5) circle (.1);
\draw[semithick] (.9, 4.8) -- ++(.1, .1) -- ++(.1, -.1);
\node at (1.1, 4.7) (developer) {};
\draw[semithick] (1, 4.9) -- ++(0, -.25);
\draw[semithick] (.9, 4.55) -- ++(.1, .1) -- ++(.1, -.1);
\node at (.7, 4.3) {sviluppatore};

\node[draw, ellipse] (uc1) at (6, 5.6)  {UC2.1 visualizza offerte da eseguire};
\node[draw, ellipse] (uc2) at (6, 4.8)  {UC2.2 visualizza ordini};
\node[draw, ellipse] (uc3) at (6, 4)  {UC2.3 visualizza dettagli offerta};
\node[draw, ellipse] (uc4) at (6, 3.2)  {UC2.4 inserisci preventivo};
\node[draw, ellipse] (uc5) at (6, 2.4)  {UC2.5 aggiorna dettagli offerta};

\draw(developer.east) -- (uc1.west)
  (developer.east) -- (uc2.west)
  (developer.east) -- (uc3.west)
  (developer.east) -- (uc4.west)
  (developer.east) -- (uc5.west);

\draw (2.5, 1.9) rectangle ++(7, 4.4) node[below left] {ERP-1};

\end{tikzpicture}

  \caption{Operazioni svolte dallo sviluppatore (UC2)}
\end{figure}



\end{document}
