\documentclass[compress,9pt]{beamer}
\mode<presentation>
%*******************************************************************************
% Codifica e lingua
%*******************************************************************************
\usepackage[T1]{fontenc}
\usepackage[utf8]{inputenc}
\usepackage[english,italian]{babel}

%*******************************************************************************
% Carattere
%*******************************************************************************
\usepackage{lmodern}

%*******************************************************************************
% Le solite macro
%*******************************************************************************
\newcommand{\team}{\textsf{Eta\,Beta\,Software}\xspace}
\newcommand{\customer}{\textsf{Alpha\,\&\,Partners}\xspace}
\newcommand{\inglese}[1]{\foreignlanguage{english}{#1}}
\newcommand{\tick}{\textcolor{green}{\ding{52}}}
\newcommand{\cross}{\textcolor{red}{\ding{56}}}
\newcommand{\mktg}{\foreignlanguage{english}{marketing}\xspace}
\newcommand{\sw}{\foreignlanguage{english}{software}\xspace}

%*******************************************************************************
% Tabelle
%*******************************************************************************
\usepackage{array}
\usepackage{booktabs}

%*******************************************************************************
% Immagini
%*******************************************************************************
\graphicspath{{../pictures/}}

%*******************************************************************************
% Grafica vettoriale
%*******************************************************************************
\usepackage{tikz}
\usetikzlibrary{arrows}

%*******************************************************************************
% Pacchetti utili
%*******************************************************************************
\usepackage{pifont} % per i dingbat
\usepackage{multicol} % per dividere contenuto in più colonne
\usepackage{xspace} % per spazi condizionali
\usepackage{eurosym} % per il benedetto simbolo dell'euro

%*******************************************************************************
% Tema di beamer
%*******************************************************************************
\usetheme{JuanLesPins}
\setbeamercovered{dynamic}
\usecolortheme{orchid}

%*******************************************************************************
% Un po' di metadati
%*******************************************************************************
\title{Consulenza per software BPM}
\author{\team}
\date{24 luglio 2013}

% fine del preambolo e inizio del documento
\setcounter{tocdepth}{2}
\begin{document}

%*******************************************************************************
% Titolo
%*******************************************************************************
\begin{frame}
\maketitle
\end{frame}

%*******************************************************************************
% Indice
%*******************************************************************************
\begin{frame}
\begin{multicols}{2}
\tableofcontents
\newpage
\begin{figure}
\setlength{\fboxsep}{.2pt}
\rotatebox{-10}{\fbox{\includegraphics[width=.4\textwidth]{logo}}}
\end{figure}
\end{multicols}
\end{frame}

\section{Progetto di consulenza}

\subsection{Introduzione}
\begin{frame}% #0
\frametitle{Prospettiva}
\begin{columns}
\column{.49\textwidth}{
\begin{figure}
  \centering
  \includegraphics[width=.7\textwidth]{alpha}
\end{figure}
\begin{block}{Informazioni:}
  \begin{itemize}
    \item \mktg tradizionale \& digitale
    \item 2 dipendenti $+$ titolare
    \item fatturato di \EUR{200.000}
  \end{itemize}
\end{block}

\uncover<2->{
\begin{alertblock}{Problemi:}
  Difficoltà gestione processi aziendali:
  \begin{itemize}
    \item analisi necessità clienti
    \item predisposizione offerte
    \item negoziazione e stesura proposte
  \end{itemize}
\end{alertblock}
}
}
\column{.49\textwidth}{

\hspace{2em}
\begin{tikzpicture}
\node[inner sep=0pt] at (0, 0) {
\includegraphics[width=.6\textwidth]{postit}
};
\node[inner sep=0pt] at (.2, .1) {
\rotatebox{3}{\begin{minipage}[b]{.45\textwidth}
\scriptsize
Nel mondo l'abito\\ fa il monaco:\\ il nostro obiettivo è\\ farvi l'abito, il monaco\\ lo dovete mettere voi!
\end{minipage}
}
};
\end{tikzpicture}

\uncover<3->{
\begin{block}{Informazioni:}
  \begin{itemize}
    \item consulenza e sviluppo soluzioni \sw aziendali
    \item 6 dipendenti
  \end{itemize}
\end{block}

\vspace{-10pt}

\begin{figure}
  \phantom{\includegraphics<1>[width=.7\textwidth]{logo}}
  \includegraphics<3->[width=.7\textwidth]{logo}
\end{figure}
}
}
\end{columns}
\end{frame}

\subsection{Pianificazione}
\subsubsection{Work Breakdown Structure}
\begin{frame}% #1
\frametitle{Organizzazione del lavoro}
\end{frame}

\subsubsection{Organizational Breakdown Structure}
\begin{frame}% #2
\frametitle{Organigramma}
\end{frame}

\subsubsection{Matrice delle responsabilità}
\begin{frame}% #3
\frametitle{Matrice delle responsabilità}
\end{frame}

\subsubsection{Resource Breakdown Structure}
\begin{frame} % #4
\frametitle{Classificazione delle risorse}
\end{frame}

\subsubsection{Pianificazione temporale}
\begin{frame}% #5
\frametitle{Pianificazione temporale}
\end{frame}

\subsection{Aspetto economico}
\begin{frame}% #6
\frametitle{Aspetto economico-finanziario}
\end{frame}

\subsection{Gestione dei rischi}
\begin{frame}% #7
\frametitle{Gestione dei rischi}
\end{frame}

\section{Studio di mercato}
\subsection{Software BPM}
\begin{frame}% #8
\frametitle{I software di Business Process Management}
\end{frame}

\subsection{Analisi temporale}
\begin{frame}% #9
\frametitle{Prospettiva temporale}
\end{frame}

\subsection{Analisi geotopografica}
\begin{frame}% #10
\frametitle{Analisi geotopografica}
\end{frame}

\section{Software selezionati}
\newcommand{\progname}{Bonita BPM}
\subsection{\progname}
\begin{frame}% #11
\frametitle{\progname}
\end{frame}

\renewcommand{\progname}{ProcessMaker}
\subsection{\progname}
\begin{frame}% #12
\frametitle{\progname}
\end{frame}

\renewcommand{\progname}{Bizagi}
\subsection{\progname}
\begin{frame}% #13
\frametitle{\progname}
\end{frame}

\renewcommand{\progname}{Camunda}
\subsection{\progname}
\begin{frame}% #14
\frametitle{\progname}
\end{frame}

\subsection{Analisi comparativa}
\begin{frame}% #15
\frametitle{Analisi comparativa}
\scalebox{.7}{
\begin{tabular}{>{\sffamily}p{.6\textwidth}*{4}{>{\sffamily}c}}
\toprule
\bfseries{}Funzionalità & \bfseries{}Bonita\,BPM & \bfseries{}ProcessMaker & \bfseries{}Bizagi & \bfseries{}Camunda\\
\midrule
modellazione                                 & \tick  & \tick  & \tick      & \cross \\
modellazione tramite editor grafico          & \tick  & \tick  & \tick     & \cross \\
rispetto notazione BPMN                      & \tick  & \cross &  \tick     & \cross \\
monitoraggio dei processi                    & \tick  & \tick  &  \tick    & \tick \\
sistema di \inglese{reporting}               & \tick  & \tick  &  \tick    & \tick \\
indicatori grafici stato di avanzamento      & \cross & \tick  &   \cross   & \cross \\
ottimizzazione dei processi                  & \tick  & \cross & \cross      & \cross \\
simulazione                                  & \tick  & \cross &  \tick       & \cross \\
sistema di \inglese{reporting} simulazione   & \tick  & \cross &  \tick      & \cross \\
validazione modelli                          & \tick  & \cross &  \tick     & \cross \\
esecuzione automatizzata processi            & \tick  & \tick  &   \tick & \tick \\
integrazione con software di parti terze     & \tick  & \tick  &   \tick     & \tick \\
interfacciamento con utenti                  & \tick  & \tick  &    \tick    & \tick \\
\inglese{form} statici                       & \tick  & \tick  &    \tick    & \cross \\
\inglese{form} dinamici                      & \tick  & \tick  & \cross        & \cross \\
\inglese{form} internazionalizzabili         & \tick  & \cross &   \cross     & \cross \\
gestione separata dei ruoli                  & \tick  & \cross & \cross       & \tick \\
\bottomrule
\end{tabular}
}
\end{frame}

\section{Proposta di soluzione}
\begin{frame}% #16
\frametitle{La nostra proposta}
\end{frame}

\end{document}
